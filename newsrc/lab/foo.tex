\documentclass{report}
\setlength{\parindent}{0in}
\setlength{\parskip}{2mm}
\title{Spice language manual (v4.0, February 2002)
}
\author{Chris Dollin}
\begin{document}
\maketitle
\tableofcontents




\part{a gentle introduction to Spice}
\chapter{introduction}


This document describes the programming language Spice, which was developed,
from an idea of Dave Raggett's, by Chris Dollin and Steve Leach. The initial
design work was done in autumn 1998 and subsequent modifications made as
implementations progressed.

The original language contained stylesheets and its syntax was closer to
ECMAScript; the current language has lost stylesheets, which need more
work to be useful in a general-purpose programming language, and its syntax
has no C-like basis. The rest of the developed language remains.

Spice is {\em dynamically typed}; by default, types are run-time entities
attached to values, rather than compile-time entities attached to expressions.
This gives flexibility in program development, at the cost of reduced
up-front error detection and some run-time overhead. Type annotations
can be added to variable declarations, and the compiler will use these for
error checking and optimisation.

Spice is {\em expression oriented}; (almost) all of the constructs of the
language return values, including loops. Spice is {\em multiple valued};
expressions denote a "row" of zero or more values. The number of values an
expression denotes is not a compile-time property.

Spice is {\em higher order}; it supports functions as values, full lexical
scoping, and has a syntax for function literals (aka lambda-expressions).

Spice is {\em multiply dispatched}; functions can be defined piecemeal
("overloaded") on differing combinations of argument types, and when the
function is called, the piece with the "best fitting" formal arguments is
executed. The usual single-receiver dynamic dispatch of object-oriented
languages (Smalltalk, C++, Java) is a special case of Spice's multi-methods,
which are inspired by CLOS and ObjectClass.

Spice is {\em class neutral}; new overloadings of any generic function can appear
in any package where that function is visible, and there is no special syntax
that distinguishes the invocation of an overloaded function from a
non-overloaded one. Classes do not provide namespaces.

The remainder of this document is in two parts; an introductory guide,
intended to give a practicing programmer a taste of Spice, followed by a
semi-formal reference manual describing the language as of February 2002.

\section{notes}


{\bf author}: Chris Dollin, {\bf version}: 3.0a, {\bf date}: December 2001

{\bf to be done}: regular expressions, facet definitions, which modifiers are
marked, {\bf super} and {\bf extends}, {\bf var} and {\bf const} to have only one blob.\chapter{a gentle introduction to Spice}


In this and subsequent sections we speak as though the reader has access to a
Spice system into which they can type Spice code and have it executed. Unlucky
readers will have to take the results on trust.

\section{hello, world}


A long-standing Unix-spawned tradition is that one's first program in a new
language should be the one that prints "Hello, World" or some suitable
variant, so here it is in Spice:

\begin{quote}
\begin{verbatim}
println( "hello, world." );
\end{verbatim}
\end{quote}
Pasting this into your Spice evaluation should result in "hello, world" being
displayed somewhere obvious.

{\bf "hello, world."} is a {\em string literal} representing a sequence of characters.
Most characters can appear as themselves in a string, but specific exceptions
are all three quoting characters ({\em string quotes} {\bf "}, {\em reserved quotes} {\bf '},
and {\em symbol quotes} {\bf `}) and the {\em escape character} $\setminus$.

{\bf println} is the name of a built-in {\em procedure}. We'll see later that Spice
has several kinds of procedure, including {\em functions}, {\em methods}, and
{\em constructors}; we use the term "procedure" to refer to them all without
distinction.

The syntax {\bf F(X)} is one form of {\em procedure call}; it evaluates the procedure
{\bf F} (which in this case is easy) and the argument(s) {\bf X} (ditto) and then
{\em calls} the procedure, supplying it with the values of the arguments. The
procedure does something (such as printing its argument) and returns some
results, which we throw away.

The semicolon is a {\em statement separator}, signifying the end of one statement
and (possibly) the beginning of another. Sometimes they are not necessary, but
this introduction willput them all in.\section{one, two, buckle my shoe}


{\bf println} isn't restricted to strings; it can do numbers, too.

\begin{quote}
\begin{verbatim}
println( 1 );
println( 1_032 );
println( 40 + 2 );
\end{verbatim}
\end{quote}
Evaluating these should result in the values {\bf 1}, {\bf 1032}, and {\bf 42} being
printed. The underbar in {\bf 1\_032} is just a visual separator. Spice has the
"usual" arithmetic operators, {\bf +}, {\bf -}, {\bf *}, and {\bf /}, although they have some
extra wrinkles we'll meet later.

What's more, {\bf println} can take multiple arguments:

\begin{quote}
\begin{verbatim}
println( 1, 2, "buckle my shoe" );
\end{verbatim}
\end{quote}
which will print {\bf 1 2 buckle my shoe}. Note that the arguments are printed
preceeded by a space (this behaviour can be changed), and that there's
only one newline printed, at the end. The degenenerate case {\bf println()} with
no arguments just prints a newline.\section{variables}


You can declare variables to hold values.

\begin{quote}
\begin{verbatim}
var x = 1 + 2 + 3 + 4 + 5 + 6 + 7 + 8 + 9 + 10;
println( x );
\end{verbatim}
\end{quote}
which will print {\bf 55}. The value {\bf x} holds can be changed
by assignment:

\begin{quote}
\begin{verbatim}
x := x * 10; println( x );
\end{verbatim}
\end{quote}
will print {\bf 550}. Using the word {\bf const} instead of {\bf var}
makes an {\em unassignable} variable (rather an oxymoron, but there it is).

Variables must be declared; trying to use a name that hasn't been declared is
an error, even if you're assigning to it, unlike eg Basic or Javascript.\section{typed variables}


It's possible to specify that a variable is restricted to holding values of a
given type. For example,

\begin{quote}
\begin{verbatim}
var count: Small = 0;
someActionsHere();
count := "of Monte Christo";
\end{verbatim}
\end{quote}
declares {\bf count} as a variable of type {\bf Small}, restricted to "small" integer
values (typically in the range plus-or-minus half-a-billion) and then throws
an error when it attempts to assign a string (of type {\bf String}) to it. If a
variable is declared with a type but no initial value, then the {\em default
value} for that type is used to initialise the variable; the default value of
type {\bf Small} is {\bf 0}:

\begin{quote}
\begin{verbatim}
var anInt: Small; println( anInt );
\end{verbatim}
\end{quote}
will print {\bf 0}. If there's no type then the variable is declared to be of type
{\bf Any} (which means what it sounds like); the default value of {\bf Any} is called
{\bf absent}.\chapter{function definitions}


There's no magic about procedures that the user cannot harness. A simple
way to define your own procedure is to write a {\bf function} definition:

\begin{quote}
\begin{verbatim}
define function add( x, y ) => x + y enddefine
println( add( 1, 2 ) );
\end{verbatim}
\end{quote}
{\bf add} is defined and then  invoked in the usual way, resulting in {\bf 3} being
printed.

Note that there's no need to use a special {\bf return} statement to specify the
procedure's return value; the value of the procedure body is returned as the
procedure result, and the body here is the sum-expression {\bf x+y}.

\section{typed arguments}


The arguments to a function can be typed in the same way that variables can.

\begin{quote}
\begin{verbatim}
define function addChecked( x: Small, y: Small ) =>
    x + y
enddefine
\end{verbatim}
\end{quote}
Calls to {\bf addChecked} with non-{\bf Int} arguments will throw an error. Spice
can sometime take advantage of the type information to produce better code
or more informative error messages. Later on, we'll see how type information
can be used to write {\em polymorphic} or {\em overloaded} functions.\section{typed results}


The result of a function can also be typed.

\begin{quote}
\begin{verbatim}
define function addToInt( x, y ) returns Small =>
    x + y
enddefine
\end{verbatim}
\end{quote}
The result from {\bf addToInt} must be an {\bf Small} value; if not an error will be
thrown. Again, the Spice compiler may be able to use this information to
generate better code or error messages.\section{indefinite arguments}


How does {\bf println} manage to have arbitrarily many arguments? It uses an
{\em indefinite argument}, marked with ellipsis notation:

\begin{quote}
\begin{verbatim}
define function gather( rosebuds... ) => rosebuds enddefine
println( gather( "while", "ye", "may" ) );
\end{verbatim}
\end{quote}
{\bf rosebuds} is an indefinite argument whose value is all of the (remaining)
arguments to the function. This prints as {\bf \{while ye may\}}, which
is the way {\em array values} are printed; the arguments are turned into an array.
(We'll see more about arrays soon.)

It's possible to have one or more {\em definite} arguments before or after a final
indefinite argument:

\begin{quote}
\begin{verbatim}
define function foo( x, y... ) as
    println( x ); println( y )
enddefine
\end{verbatim}
\end{quote}
{\bf foo} needs at least one argument {\bf x} but takes arbitrarily many more, which
are gathered into {\bf y}. It then prints both {\bf x} and {\bf y}; if there are no extra
arguments, then {\bf y} is an {\em empty array}, which prints as {\bf \{\}}. {\bf println}
returns no results, so {\bf foo} returns no results.\chapter{arrays}


Spice allows arrays to be constructed using {\em array expressions}, which are
expression sequences enclosed in braces:

\begin{quote}
\begin{verbatim}
var lots = {27, "now", "is", "the", "time", 42};
    println( lots );
\end{verbatim}
\end{quote}
which will print {\bf \{27 now is the time 42\}}. Arrays can contain values of any
type. Individual elements can be extracted by indexing:

\begin{quote}
\begin{verbatim}
println( lots[2] ); lots[2] := "anguish"; println( lots );
\end{verbatim}
\end{quote}
which will print {\bf now} and {\bf \{27 anguish is the time 42\}}. Spice arrays are
based at {\bf 1}, so element {\bf 2} is the second element. The function
{\bf length} delivers the number of elements in its array argument.

\begin{quote}
\begin{verbatim}
println
        (
        length( lots ),
        arrayBase( lots ),
        arrayBase( newlots )
        );
\end{verbatim}
\end{quote}
which will print {\bf 6 0 1}.\chapter{dot and infix notation}


It's often useful and clearer to write calls to one-argument functions,
expecially those that just extract components from structured values like
arrays, using {\em dot notation}:

\begin{quote}
\begin{verbatim}
println( lots.length, lots.arrayBase, newlots.arrayBase );
\end{verbatim}
\end{quote}
is just another way of writing the previous calls. In fact you can call
procedures of {\em any} number of arguments using dot notation, because of a
general rule of the language:

\begin{quote}
\begin{verbatim}
X.f(Y) == f(X,Y) == X @f Y, and X.f == f(X) == X @f
\end{verbatim}
\end{quote}
The forms {\bf X @f Y} and {\bf X @f} are a way of writing any function call as an
infix or postfix expression, which is often useful when you have a long chain
of function applications each of which supplies the "left" argument to the
next.

The principal difference between {\bf .} and {\bf @} is that {\bf .} is much more tightly
binding -- it's the most binding infix operator -- and {\bf @} is loosely binding,
being only just tighter than the logical connectives {\bf and} and {\bf or}.

Different styles of function call are used to make it clear what's an argument to
what.

\section{dot and infix function definition}


Although the plain fucntion definition syntax we've seen already is enough,
because of the equivalence of the different notations we've described,
Spice allows you to define a function in the style in which you expect it
to be called:

\begin{quote}
\begin{verbatim}
define function x @addWithAt y as x + y enddefine
define function x.addWithDot(y) as x + y enddefine
\end{verbatim}
\end{quote}
The arguments can still be typed using {\bf :}, and for a function
defined with dot-notation, the special {\bf on} syntax:

\begin{quote}
\begin{verbatim}
define function x:Small @addWithAtColon y:Small as
    x + y
enddefine
define function on Small x.addWithDotColon(y:Small) as
    x + y
enddefine
\end{verbatim}
\end{quote}
All four of these definitions define functions which add two values together,
the second pair demanding that those values be {\bf Int}s, and all of them can be
called in all the different ways we have discussed.

We'll see later how these notations tie into the more common object-oriented
approaches, and why the {\bf on} syntax is so-called.\chapter{ifs and buts}


So far, all of our expressions have been unconditional -- get the arguments,
do something, deliver the result; no choices are involved.

Spice has {\em conditional expressions} for expression choices, and {\em relational
expressions} for tests.

\begin{quote}
\begin{verbatim}
if lots.length < 10 then "short" else "long" endif.println;
\end{verbatim}
\end{quote}
The relational expression {\bf lots.length $<$ 10} compares the length of {\bf lots}
(which is probably still {\bf 6}) with {\bf 10} and delivers {\bf true} if it's less than
and {\bf false} if it's equal or greater. {\bf true} and {\bf false} are the built-in
values of type {\bf Bool}. Spice also has the obvious {\bf $>$}, {\bf $<$=}, and {\bf $>$=}, and {\bf =}
for equality, and the less obvious {\bf /=} for inequality.

The conditional expression tests the boolean value; if it is {\bf true} its value
is its {\bf then} arm, if it is {\bf false} its value is its {\bf else} arm, and if it's
not boolean, an error is thrown. (Unlike various other languages, {\bf 0}, the
null string {\bf ""}, and the null reference {\bf absent}, are {\em not}
alternatives to {\bf false}, and values such as {\bf 1}, {\bf "yes"}, and {\bf println} are
{\em not} alternatives to {\bf true}.)

{\bf println} is being called here with dot notation, with its argument being the
result of the {\bf if}-expression, here the string {\bf short}.

\section{elselessness and elseif}


The {\bf else} part of a conditional is optional. If you miss it out, it's as
though you'd written one with no expression following. Where you don't want a
result from the {\bf if}-expression, there's no problem; if you do, it will
usually generate the value {\bf absent}. (See the reference section for more
details.)

If you want to write a chain of tests, rather than ending up with
ever-more-deply nested {\bf if}-{\bf endif} structures, Spice lets you introduce
additional tests with {\bf elseif}:

\begin{quote}
\begin{verbatim}
if x == "hello" then
    println( "well hello there!" )
elseif x == "goodbye" then
    println( "sorry to see you go" )
else
    println( "eh? what does", x, "mean?" )
endif
\end{verbatim}
\end{quote}
You can have as many {\bf elseif}s as you need.\section{ands and ors}


The test of an {\bf if} can be composed using the special operators {\bf and} and
{\bf or}. {\bf X or Y} is true if {\bf X} is or if {\bf Y} is, but {\bf Y} is only evaluated if
{\bf X} is {\bf false}. Similarly {\bf X and Y} is {\bf true} if both {\bf X} and {\bf Y} are {\bf true},
but {\bf Y} is not evaluated if {\bf X} is false.

That's what makes these operators "special": their right operand is only
evaluated if its value is needed.\section{short conditionals}


Spice also permits "short" conditional expressions, for which it uses the same
syntax as Algol68 (by coincidence): the expression {\bf (X $\mid$ T $\mid$ F)} is
equivalent to {\bf if X then T else F endif}.\section{unless}


Spice has an alternate form of {\bf if}, the {\bf unless} (with closing keyword
{\bf endunless}) which is the same except the sense of the (first) test is
inverted.\chapter{loops}


Spice has a rich {\bf for} loop construct, which allows iteration over
collections, termination on conditions, and specification of loop results.
A subset of those is available in the {\bf repeat} construct.

\section{repeat}


Suppose we want to iterate over the elements of an array {\bf lots}.
Here's one way to do it with {\bf repeat while}:

\begin{quote}
\begin{verbatim}
var i = 0;
repeat while i < lots.length do
        ;;; stuff involving lots[i] ...;
        i +:= 1
    endrepeat
\end{verbatim}
\end{quote}
The expression {\bf i +:= 1} is equivalent to {\bf i := i + 1}, but more compact,
especially when {\bf i} is some complex expression.

For example, to find the element of an array containing a specific element by
complete linear search:

\begin{quote}
\begin{verbatim}
define function findIndex( a, x ) as
    var i = 0;
    var here = 0;
    repeat while i < a.length and here = 0 do
        if a[i] = x then here := i endif;
        i +:= 1
    endrepeat;
    here
enddefine
\end{verbatim}
\end{quote}
The index of the first element equal to {\bf x} is stored into {\bf here} and then
returned as the procedure result. The equality test {\bf =} compares values
using type-specific code.

If you're wondering why Spice uses {\bf repeat while}, and not just {\bf while}, the
reason is that it also has {\bf repeat until} (where the condition is inverted),
and that you can write code between the {\bf repeat} and the {\bf while}. Also, as
we'll see below, {\bf while} is also used in {\bf for} loops.\section{for}


We can perform the same loop with automatic counting and implicit value
return, using a {\bf for}-loop with two {\em control clauses}:

\begin{quote}
\begin{verbatim}
define function findIndex( a, x ) as
    for
        i from 1 to a.length
        until a[i] = x then i
        finally 0
    endfor;
enddefine
\end{verbatim}
\end{quote}
The variable {\bf i} is automatically declared as a varible that exists only while
the loop is executing; it takes values from the {\bf from} value (here, {\bf 1}) to
the {\bf to} value (here, {\bf a.length}) inclusive. The loop terminates "early" when
an element {\bf a{[}i{]} = x} is found, and it delivers the result {\bf i}. If the loop
terminates "normally", it returns the value of the "finally" clause, here
{\bf 0}.

{\bf for} loops can also iterate over collections, using the {\bf in} syntax:

\begin{quote}
\begin{verbatim}
define function findIndex( a, x ) as
    for
        i from 1; ax in a
        until ax == x then i
        finally 0
    endfor;
enddefine
\end{verbatim}
\end{quote}
{\bf ax} is automatically declared, and takes on the value of each element in
turn; we still need to keep track of the index, so we count {\bf i} from 1
upwards anyway.\chapter{multiple values}


So far, we've seen Spice expressions and procedures which deliver {\em single}
results. Spice expressions can deliver {\em multiple} results, often avoiding the
need to construct new objects or to use "reference parameters" (which Spice
hasn't got).

\begin{quote}
\begin{verbatim}
define function plusOrMinus( x, y ) =>
    (x + y, x - y)
enddefine
\end{verbatim}
\end{quote}
The comma keeps both its left and right operand values (exactly as it does in
a function argument list); {\bf plusOrMinus} delivers {\em two} values.

\begin{quote}
\begin{verbatim}
println( 20 @plusOrMinus 17 );
\end{verbatim}
\end{quote}
will print {\bf 37 3}.

\section{void context}


Some expressions, in particular top-level expressions, are evaluated in {\em void
context}; they are evaluated and their results thrown away. The Spice system
you are using to test our examples runs them in void context.

Sometimes you need to make it explicit that, whatever else would have
happened, you want no values to escape. The expression {\bf none E} evaluates E in
a new multiple-value context and then throws away all the answers, delivering
none. The expression {\bf none} is a shorthand for {\bf none ()}, representing no
values. Similarly, the expression {\bf one E} evaluates {\bf E} and turns it into one
value: if {\bf E} delivered no values, {\bf one} returns {\bf absent}; if {\bf E} delivers one
value, {\bf one} returns that; and if {\bf E} delivers several values, {\bf one} returns
the first one.\section{multiple-value loops}


The result of a loop is all the results from the executions of its body,
plus any finally-values.

\begin{quote}
\begin{verbatim}
println( for i from 1 to 10 do i*i endfor )
\end{verbatim}
\end{quote}
to print {\bf 1 2 9 16 25 36 49 64 81 100}.\section{explode}


One built-in function produces multiple results as a matter of course;
{\bf explode}. {\bf explode} takes one argument and explodes it into its constituents
(if any) as a multiple result. Applied to an array, it explodes it into its
elements; applied to a string, it explodes it into its characters. Applied
to an atomic object (a number, a character, a boolean, {\bf absent}) it delivers
{\bf none}.\chapter{procedures as values}


Spice procedures are not restricted to being defined and called; there are
several ways they can be manipulated.

\section{first-class procedures}


Spice procedures are {\em first-class values}; they can be passed as parameters,
returned as results, and stored into variables and data-structures (such as
arrays; we'll see more later).

This is particularly useful when you need to do something to all the elements
of a collection, but the details of how the collection are kept is secret
(eg to allow you to change it later); you write a function that walks over
the collection and does "something" to each element. Since all we've got
is arrays, that forms our first example; let's suppose that {\bf them} is
a secret array variable.

\begin{quote}
\begin{verbatim}
define function appCollection( f ) as
    for x in them: Array do x.f endfor
enddefine
\end{verbatim}
\end{quote}
Now the expression {\bf appCollection(println)} (which of course can also be
written {\bf println.appCollection}) will print all the elements of {\bf them}. If the
elements {\em do} something sensible when passed to some function {\bf mangle}, then
{\bf mangle.appCollection} will mangle all of them.

Note that {\bf appCollection} returns all the values that {\bf f} returns for the
caller to do with as they like. Of course, we might want to construct a new
collection from the old one by mangling each element:

\begin{quote}
\begin{verbatim}
define function mapCollection( f ) =>
    {f.appCollection}
enddefine
\end{verbatim}
\end{quote}
which gathers up the results from {\bf appCollection} and makes a new collection
(here, array) out of them.\section{lambda-expressions}


Passing functions as arguments, and applying them, is all very well, but it
can get tedious defining trivial functions to use. For example, if
we want to make a new collection by incrementing every element of the
old one, we can write:

\begin{quote}
\begin{verbatim}
define function inc( x ) as x + 1 enddefine
inc.mapCollection.println;
\end{verbatim}
\end{quote}
If {\bf them} is, say, the collection {\bf {[}0 8 41{]}} then this will print {\bf {[}1 9 42{]}}.
But it's tedious to have to define a function, put it
somewhere in the code, give it a suitably mnenomic (but short) name, then then
pass it. Wouldn't it be nice if you could write the function right where you
were going to pass it?

Of course the answer is "yes", otherwise we wouldn't have posed the question.
The expression {\bf (A =$>$ E)}, where {\bf A} is an argument list (with the brackets
omitted) and {\bf E} is a sequence of expressions, is called a {\bf lambda expression}
(for historical reasons to do with Church's {\em lambda calculus}) or a {\em procedure
literal}. Lambda-expressions are intended to be used for {\em short} procedures;
otherwise they can make the code look cluttered.

Instead of defining {\bf inc} above, we can write {\bf (x =$>$ x +
1).mapCollection.println} for the same effect. To double each element, use
{\bf (x =$>$ x * 2)}; to square it, use {\bf (x =$>$ x * x)}; to replace it with {\bf 0}
use {\bf (x =$>$ 0)}.

\subsection{"full lexical scope"}


Because we haven't said you can't, the reader may suppose that you can write
lambda-expressions within procedures, and similarly, that those expressions
may refer to, even assign to, local variables of those procedures. And you
can.

\begin{quote}
\begin{verbatim}
define function konst( x ) as (ignored as x) enddefine
\end{verbatim}
\end{quote}
{\bf konst} (so-called because {\bf const} is a Spice reserved word) is a function
that makes constant functions; {\bf konst(E)} is a function that delievrs the
value of {\bf E} whatever it's applied to, so {\bf konst(0)} is a function {[}of one
argument{]} that always delivers {\bf 0}. The variable {\bf x} which is the parameter to
{\bf konst} is available to the lambda-expression even after {\bf konst} has finished
executing.

This behaviour (that the variables of a function are visible to
lambda-expressions within that function, and that they live on after the
function exits)is called {\em full lexical scoping}. It's relatively uncommon
in programming languages (with the notable exceptions of Common Lisp, Scheme,
Pop11, and Smalltalk), but it is your friend.\section{hole-expressions}


Lots of little lambda-expressions are just an operator (or function call)
with one (or two) arguments "missing"; like {\bf inc} and its brethren above.
For these Spice allows you to write a special form of lambda-expression, the
{\em hole expression}, which is a function call or operator invocation with
some of its arguments replaced by {\em holes}. The usual hole is written {\bf ?},
and it stands for "the {[}first{]} argument"; it makes the application or
invocation into a lambda expression.

So {\bf (? + 1)} (the brackets are unnecessary) is another way of writing {\bf (x =$>$
x + 1)}, where {\bf x} has been replaced by the hole and the argument declaration
is unnecessary.

{\bf (? - 1)} is the function that subtracts {\bf 1} from things, {\bf (42 - ?)} is
the function that subtracts things from {\bf 42}, and {\bf (? * ?)} is the
function that squares things.

Just in case you ever want to use holes for functions with more than one
argument, there are as many holes as you like, written ?{\bf 1}, ?{\bf 2}, and so
on; {\bf ?} is shorthand for {\bf ?1}. A hole-expression has as many implied
arguments as the biggest hole it uses; the Spice compiler will be curious
about expressions with missing holes (eg {\bf ?19 + 1}).\section{combining functions}


Because it's possible to write functions that take and deliver other
functions, its possible to write {\em combining functions} who's job is just
to manipulate other functions; this can result in a style known as {\em higher
order programming}.

The standard function {\bf Then} {[}note the capital T!{]} takes two functions
{\bf f} and {\bf g} and returns a new function that takes some arguments {\bf X},
applies {\bf f} to them, and then applies {\bf g} to the result; it might be written
as:

\begin{quote}
\begin{verbatim}
define function Then( f: Function, g: Function ): Function as
    (args... as args.explode.f.g)
enddefine
\end{verbatim}
\end{quote}
Note that {\bf explode} generates arbitrarily many results, all of which become
arguments to {\bf f}; all of {\bf f}s results become arguments to {\bf g}; and all
of {\bf g}s results become the results of the lambda-expression. Nothing
special has to be done by the programmer to keep tabs of how many arguments
and results are returned.\chapter{more on {\bf absent}}


It turns out that code like "the value of X, unless it's absent, in which case
the value of Y" appears quite often, for example, doing table-lookups. Spice
allows this to be written compactly using the operator {\bf $\mid$$\mid$}.

{\bf X $\mid$$\mid$ Y} has the value {\bf X}, unless it's {\bf absent}, in which case it has the
value {\bf Y}; {\bf Y} is not evaluated unless it has to be ({\em ie}, {\bf X} is {\bf absent}).
It's rather like a pumped-up {\bf or} where any non-absent value counts as {\bf true}.

Similarly, {\bf X \&\& Y} is {\bf absent} if {\bf X} is {\bf absent}, and the value of {\bf Y}
otherwise; again, {\bf Y} is only evaluated if it has to be ({\em ie}, {\bf X} isn't
{\bf absent}). It's rather like a pumped-up {\bf and}.\chapter{classes}


So far we've only uses (some) of the built-in Spice types: integers, strings,
arrays, and procedures. Spice also allows you to define your own data-types
using {\em classes}.

\section{class definition}


A class definition allows you to define a new (kind of) {\em object} which you
can use in your code. The simplest class definition introduces an object
with named slots:

\begin{quote}
\begin{verbatim}
define class Pair
    slot front: Any = absent
    slot back: Any = absent
enddefine
\end{verbatim}
\end{quote}
This defines a new class object, {\bf Pair}, with two slots called {\bf front} and
{\bf back}. The {\bf slot} declaration is just like a {\bf var} declaration, except that
it makes variables "inside" objects; you can leave out the initialiser and
type in the usual way.

The name {\bf Pair} becomes a new type name, and you can use it in the type part
of a variable declaration or procedure argument.

You can print out the class if you wish; {\bf Pair.println} will print something
like {\bf $<$Class Pair extends $<$absent$>$$>$}. (We say "something like" because it's
possible to change the default way objects get printed.)

You can make new {\bf Pair} objects with the expression {\bf new Pair()}. Each such
object has a {\bf front} and a {\bf back}, which start off as {\bf absent}, and which you
can change. Every object also responds to the function {\bf typeof} by delivering
its class object.\section{updaters}


Suppose we've declared {\bf var p = new Pair();} so that {\bf p} holds a new {\bf Pair}
object. We can assign to the {\bf front} and {\bf back} slots of {\bf p}:

\begin{quote}
\begin{verbatim}
p.front := 1, p.back := "two";
    p.front.println, p.back.println;
\end{verbatim}
\end{quote}
prints {\bf 1 two}. In Spice, it's possible to call procedures
on the left-hand-side of an assignment, in which case what happens is that
the {\em updater} of that procedure is called. Just as Spice makes procedures
to access the {\bf slot}-slots of a class value, it gives them updaters
to change those values.

\begin{quote}{\em technical note}. Unlike some languages (Common Lisp, Dylan), it is the
procedure {\em value} that has the updater, not the procedure {\em name}. This means
that you can call the updaters of procedures passed as parameters or stored
into data structures. We stole/////re-used this behaviour from Pop11.
\end{quote}You can define your own updaters if you wish:

\begin{quote}
\begin{verbatim}
define function x.foo := y as
    println( "updating", x, "with", y )
enddefine
\end{verbatim}
\end{quote}
The {\bf := y} says this this is defining the updater of {\bf foo} (and there had
better be a definition for {\bf foo} elsewhere; you can't define the updater of a
procedure that doesn't exist) and that {\bf y} is the name of the parameter which
is the "right-hand-side of the assignment". {\bf y} can be typed, as usual.

While an updater can do whatever it likes, it is very stongly advised that
updaters are given only to access functions, and that they do the "obvious
thing"; after an assignment {\bf x.foo = E} then the expression {\bf x.foo} should
deliver the value {\bf E} that was assigned.

If {\bf foo} is a procedure, then {\bf foo.updater} is the updater of that procedure,
if it has one, and {\bf absent} otherwise.
\section{methods}


A class can also define {\em methods} which act on objects of the type it
defines.

\begin{quote}
\begin{verbatim}
define class Pair
#
    slot front: Any = absent
#
    slot back: Any = absent
#
    define method wipe() as
        this.front = absent; this.back = absent; none
    enddefine
#
    define method setPair(x,y) as
        this.front = x; this.back = y; none
    enddefine
#
enddefine
\end{verbatim}
\end{quote}
This version of {\bf Pair} defines two methods, {\bf wipe} and {\bf setPair}. {\bf wipe} sets
both slots to {\bf absent}, and {\bf setPair} sets them to the given parameter values.
The name {\bf this} is bound to an implicit additional parameter, which is the
instance of {\bf Pair} to be changed.

The expression {\bf none} delivers no values; there doesn't seem to be a sensible
value to deliver from either of these methods.

Methods are procedures, and you call them in the usual ways; if {\bf p} is an
instance of this {\bf Pair}, then {\bf p.wipe} will wipe it, and {\bf p.setPair(1,2)} will
set its {\bf front} to {\bf 1} and its {\bf back} to {\bf 2}. You can tell from this that
{\bf this} is bound to the {\em first} argument value. In fact, if it takes your
fancy, you can use a different name than {\bf this}, using the same
function-definition syntax that we saw earlier:

\begin{quote}
\begin{verbatim}
define method self.wipe as
    self.front = absent; self.back = absent; none
enddefine
\end{verbatim}
\end{quote}
would be a suitable replacement for {\bf Pair}'s version of {\bf wipe}.

We'll see more about methods when we discuss inheritance, later.\section{initialisers}


The {\bf new Pair()} syntax is a bit limited, because you can't easily set new
values for its slots. If you want to create a new object {\em and} set the
values of its slots, you cant define an {\em initialiser}. Here, {\bf Pair} gets a
third outing:

\begin{quote}
\begin{verbatim}
define class Pair
    slot front: Any = absent
    slot back: Any = absent
    define init pair(x,y) as
        this.front := x, this.back := y
    enddefine
enddefine
\end{verbatim}
\end{quote}
You can make a new initialised {\bf Pair} by writing {\bf new pair("hello","world")}.
This works by making a new {\bf Pair} and then handing that on as the {\bf this} for
{\bf pair}. Any values returned by an initialiser are discarded; the result of a
{\bf new} expression is the newly constructed object.

You can have as many initialiser definitions as you like for a class. As a
special favour, you can use {\bf define init} (with no name) inside a definition
of a class {\bf P} to define overloadings of {\bf new P(X)}.

\begin{quote}{\em Aside}. We've now met almost all the types of procedure you can have in
Spice; functions, methods, initialises, and the different flavours of
lambda-expressions. There are {\em predicates} which can tell them apart, although
you will usually only need these when you're debugging code.\begin{itemize}\item {\bf x.isProcedure} is {\bf true} when {\bf x} is any kind of procedure, and {\bf false}
otherwise.

\item {\bf x.isMethod} is {\bf true} if {\bf x} is defined by a {\bf method} definition, and {\bf false}
otherwise;

\item {\bf x.isFunction} if {\bf x} is defined by a {\bf function} definition, and {\bf false}
otherwise;

\item {\bf x.isInitialiser} is {\bf true} when {\bf x} has been defined with {\bf method new}, and
{\bf false} otherwise;

\item and {\bf x.isLambda} is {\bf true} when {\bf x} is the result of a lambda-expression or
hole-expression, and {\bf false} otherwise.

\end{itemize}\end{quote}\section{inheritance}


A class can {\em extend} an existing class. Here's one example; we might
choose to make triples an extension of pairs.

\begin{quote}
\begin{verbatim}
define class Triple extends Pair
    slot side
    define init triple(x, y, z) extends pair(x, y) as
        this.side := z
    enddefine
endclass
\end{verbatim}
\end{quote}
{\bf Triple} has all the slots that {\bf Pair} does, and one more: {\bf side}. {\bf front} and
{\bf back} automatically work on {\bf Triple}s. {\bf Triple}s {\em are} {\bf Pair}s, and then
some. The initialiser {\bf triple} works by first invoking the initialiser {\bf pair}
in its {\bf extends} clause to do the first two elements, and then assigning the
third itself.

Not just {\bf front} and {\bf back}, but {\em any} function that works on {\bf Pair}s and
doesn't explicitly exclude non-{\bf Pair}s will work on {\bf Triple}s.\section{overloading functions}


How does this happy state of affairs come about? When the argument type is the
name of a class (like {\bf Pair}), then it will accept values of any of its
extensions (like {\bf Triple}).

Sometimes, we want the behaviour of a method to be different in sub-classes
than in the parent. In this case, the parent must define the method to be
{\bf generic}, and the different definitions for the sub-classes must be declared
as {\bf specific}. This works outside as well as inside classes; the only
difference is that inside a class, the implicit {\bf this} argument is added to
the method signature.

Here's an example of generic procedures outside a class:

\begin{quote}
\begin{verbatim}
define generic size( x: Any ) as 0 enddefine
define specific size( x: Small ) as x @logToBase 2 enddefine
define specific size( x: String ) as x.length enddefine
\end{verbatim}
\end{quote}
This defines the {\bf size} of an integer to be its bitswidth (and blows up if {\bf x}
is {\bf 0}), the {\bf size} of a string to be its length, and the size of anything
else to be {\bf 0}. The choice of which definition of {\bf size} to use is made when
{\bf size} is called, by looking at the types of its arguments.\chapter{type-expressions}


So far we've said little about the type-expressions that can be written
following {\bf :}, except that certain built-in names ({\bf Any}, {\bf Small}, {\bf String})
are allowed, and the names of any classes.

In fact types can be expressed with {\em expressions}, which have the same syntax
and semantics as value-expressions. There are four important type-expressions
in Spice: {\em multiple}, {\em lots}, {\em array}, and {\em non-optional} types.

\section{multiple}


If {\bf T} and {\bf U} are types, then {\bf T, U} is the {\em multiple type} which is a value
of type {\bf T} then a value of type {\bf U}.\section{lots}


If {\bf T} is a type, {\bf T**} is the {\em lots of T} type, which is some unspecified
number of {\bf T}s as multiple values. The return type of a procedure defaults to
being {\bf Any**}.\section{arrays (and based arrays)}


If {\bf T} is a type, then {\bf \{T\}} is the type {\em array of T} or {\em T row}. Note
carefully that the Spice array constructor {\bf \{E\}} {\em always} makes values of
type {\bf \{Any\}}, because Spice arrays can hold any kind of value; to make eg
an {\bf \{Small\}} you have to invoke a special constructor {\bf IntArray(E)}.\section{optional types}


Unless otherwise specified, {\bf absent} is not permitted where a value of a
specific (non-{\bf Any}) type is expected; so a variable typed as {\bf Small} cannot
legally be given {\bf absent} as its value. This can be over-ridden by using the
{\bf optional} type constructor; if {\bf T} is a type, {\bf T??} is the type which all the
values of type {\bf T} and also the value {\bf absent}.\chapter{packages}


So far we've seen fragments of Spice code out of context for illustrative
purposes. In real life, Spice code code should be organised into {\em packages}.
A package is a collection of procedures, variables, and classes (and styles,
which we'll see later) which work together to provide a coherent service to
the programmer who uses those packages.

A package starts with a {\em package header}, which identifies the package and the
version of Spice it was intended for, specifies any other packages it may need,
and then continues with the {\em package body}, which is a sequence of variable,
procedure, class, and other declarations, or the kind we've discussed
elsewhere.

The header starts with the {\em spice specification part}, which consists of
the reserved word {\bf spice}, a version string, and possibly some {\em preference
settings}. The version string is a string specifying the version of Spice
that the package is supposed to work with, and the preference settings control
the Spice compiler -- see later for details.

Otherwise the header starts with the reserved word {\bf package} and the package
name, which is a series of simple names deparated by dots. For example,

\begin{quote}
\begin{verbatim}
spice "release.2.0";
package this.is.an.example;
\end{verbatim}
\end{quote}
Spice does not constrain package names to conform to file names on your local
system, but it's a good idea to put a package whose name ends in {\bf .foo} in a
file called {\bf foo}.

\begin{quote}{\em technical note I}. An IDE for Spice is {\em required} not to insist that package
names form a path through a filing system hierarchy, and is {\em required} to
accept package name components longer that the local filing system component
names, and with "strange" characters in.{\em technical note II}. The current Spice prototype compiler allows the mappings
from package names to filestore names to be specified in package mapping
entries in its initialisation file, as part of a proof-of-concept.
\end{quote}\section{public and private}


A package exists to make a service available to its users, which it does by
making some of its identifiers visible to those users. By default, {\em none} of
the identifiers in a package are visible. However, declarations can be marked
as {\em public} by writing {\em declaration qualifiers} in square brackets after
the introductory keyword. To make a public variable, for example:

\begin{quote}
\begin{verbatim}
public var exposed: String = "hello"
\end{verbatim}
\end{quote}
declares a public string-only variable called {\bf exposed}. You can mark a
variable {\bf private} instead, but since this is the default, it won't make much
difference.

You can declare a function {\bf public} in the same way:

\begin{quote}
\begin{verbatim}
define public function inc( x ) as x + 1 enddefine
\end{verbatim}
\end{quote}
More importantly, you can declare a {\bf class} public. If you do so, the
class name is public, {\em and so are all the non-slot methods declared in
it}. The slot methods (ie the names of {\bf slot}-variables) are still private by
default; you can declare them public explicitly, and you can declare the other
methods private explicitly.\section{imports}


{\bf public} is one side of the coin; {\bf imports} is the other. A package can
{\em import} another one in its header.

\begin{quote}
\begin{verbatim}
package using.example;
import this.is.an.example;
\end{verbatim}
\end{quote}
This makes all the public identifiers from {\bf this.is.an.example} visible inside
{\bf using.example}. If you don't specify where this other package is to be found,
the Spice compiler will use a set of rules to locate it; but you can force its
hand:

\begin{quote}
\begin{verbatim}
import this.is.an.example from "/home/hedgehog/example.spi";
import this.is.an.example from
    "http://cdollin/Spice/Modules/example.spi";
\end{verbatim}
\end{quote}
Following the {\bf from} is an expression (almost always a string literal) with
the URL for the source of the package.

Spice does not require than imported packages are written in Spice; it is
expected that implementations will define ways to import packages written in
commonly-used languages such as C, C++, Java, and Javascript.\section{readonly}


It's possible to qualify public variables and slots as {\em read only}, which
means that they can be read but not changed by importing packages. For ordinary
variables, all that need be done is to prohibit assignment to them; but
for slots, which are accessed by procedures, it's not quite so simple.

When a slot is marked read-only

\begin{quote}
\begin{verbatim}
public readonly slot magic = 42;
\end{verbatim}
\end{quote}
({\bf readonly} is not a reserved word) then the compiler makes {\em two} versions of
{\bf magic}; one for local use, and one for export. {\em Only the local version has an
updater}. This can cause the interesting situation:

\begin{quote}
\begin{verbatim}
package defines.magic;
#
define class P
    public readonly slot magic = 72;
enddefine
#
define function isEqual( x ) as
    println( x == magic )
enddefine
#
endpackage
package Q
#
import defines.magic;
#
testEqual( magic );
#
endpackage
\end{verbatim}
\end{quote}
which will print {\bf false} as the exported {\bf magic} is a different function to the
local one.\section{protected access and aliases}


It's possible to import a package, but to declare that you need to specify its
name everywhere you specify its variables:

\begin{quote}
\begin{verbatim}
import protected defines.magic;
\end{verbatim}
\end{quote}
To refer to {\bf magic} now you need to prefix it with the last part if its
package name: {\bf magic::magic}. (Requiring the {\em entire} package name would be
overkill; package names are supposed to be long and descriptive.)

The main use of {\bf protected} is to allow an imported package to define lots of
useful names (usually constants) without polluting your own packages
namespaces.

If you don't like the last part of the package name, or if it would be
ambiguous, you can change the part you specify:

\begin{quote}
\begin{verbatim}
import protected boo = defines.magic;
\end{verbatim}
\end{quote}
Now {\bf boo} is an alias for {\bf defines.magic} and we can write {\bf boo::magic}.\section{pervasive import}


A package that imports some identifiers -- say, {\bf brick} and {\bf concrete} -- from
another package does not by default re-export them; they're not part of its
{\em public interface}. They can be injected into the public interface by
qualifying the import with {\bf public}:

\begin{quote}
\begin{verbatim}
public import defines.magic;
\end{verbatim}
\end{quote}
Now this package exports {\bf magic::magic}.\chapter{switches}


The power of Spice's overloaded functions means that you won't write switches
so often as you might in a more conventional language. But you still need to
be able to choose from finite selections of integer and string values, so
Spice has a general {\bf switch} construct:

\begin{quote}
\begin{verbatim}
switch x into
    case 1: statement1();
    case 2, 3: statement2();
    case "hello":
    case "world": statement3();
    default: statement4();
endswitch
\end{verbatim}
\end{quote}
If {\bf x} has the value {\bf 1}, then {\bf statement1()} is executed. If it has the value
{\bf 2} or {\bf 3}, then {\bf statement2()} is executed. If it has the value {\bf "hello"} or
{\bf "world"}, {\bf statement3()} is executed. And if it has none of these values,
{\bf statement4()} is executed.

Note that the values in the switch can be integers or strings (or symbols);
indeed the {\bf case} values need not be compile-time constants. (However, the
switch construct is more efficient with compile-time constants, and more
efficient with integers than with strings and symbols.) If a switch has a
compile-time case label of {\bf N} then that takes priority over any run-time case
label evaluating to {\bf N}.

Also note that there's no need for a {\bf break} or similar construct to leave the
{\bf switch}; the case body starts with its first statement and ends just before
the next {\bf case} or {\bf default}. (You can, of course, only have one
{\bf default} per {\bf switch}).

If there's no {\bf default}, it's as though {\bf default: none;} had been written.
\chapter{numeric literals}


So far we've taken numbers pretty much for granted, writing them as digit
sequences with the occasional {\bf \_} for visual clarity. Spice has a rather
extensive set of {\em numeric literals}.

All numeric literals start with a digit, and all numeric literals may
contain underbars (which are ignored; they are there for the reader's visual
convenience). When numeric literals contain letters, their case is
irrelevant.

A numeric literal may begin with a {\em base specifier}, which is a series of
digits ending with the letter "x". The digits, taken as a decimal number,
specify the base of the numeric literal; the base must be in the range
{\bf 2..36}. If the base is {\bf 0} it is taken to mean base {\bf 16} for compatability
with {\bf C} (etc). The extended digits are "a" (for {\bf 10}) up to "z" (for {\bf 35}).
Using digits too large for the base is an error.

A numeric literal that has a base less than or equal to {\bf 10} may end with
a series of letters specifying the {\bf units} in which the value is expressed.
Typical units are {\bf px} (pixels), {\bf in} (inches), {\bf s} (seconds), {\bf em} (ems -- a
printing unit), and {\bf mm} (millimetres).

\begin{quote}To get values with large bases to have units, you need to use a different
way of expressing those units, eg using function calls.
\end{quote}A numeric literal may have a {\em scaling factor}, which is the letter "e",
optionally followed by a sign ("+" or "-") and a series of digits. The digits
represent a number in decimal, and the scaling factor is the literal's base
raised to that power. If the literal's base is greater than {\bf 10} then the sign
is not optional (otherwise {\bf 0x4e5} would be ambiguous). If present, the
scaling factor is before the units.

An {\em integral} literal is a series of digits, possibly with a base specifier
(in which case the digits can include letters), possibly with a scaling factor
and units. A {\em floating} literal is a series of digits, a decimal point, and a
second series of digits; again, it may include a scaling factor and units.\chapter{units}


To support its use in laying out Web pages (and for other graphic operations),
Spice values can have {\em units}. Several units are built-in to the language,
notably {\em inches} ({\bf in}), {\em millimeters} ({\bf mm}), {\em points} ({\bf pt}), and {\em pixels}
({\bf px}).

Writing a unit literal is easy, as described earlier; {\bf 4in} is four inches,
{\bf 27mm} is twenty-seven millimetres, {\bf 100px} is one hundred pixels. These
pre-defined names are also available as postfix operators, so {\bf x in} and
{\bf (y + 1)px} are legal, and by special dispensation of the dot operator can
appear as functions using dot, so {\bf x.pt} is also legal.

The arithmetic operators work on values with units in the "natural" way,
so you can add and subtract values with like units. (You can't just add a
number to a unit value, though; {\bf 42in + 1} will generate a run-time error.)
What's more, if the units are different but have the same underlying
dimension (eg both lengths), then they can still be added and subtracted, by
converting them to a common scale.

You can multiply values with units by numbers, in the obvious way. You can
multiply values with units together, too; {\bf 6in * 7in} is {\bf 42 in in}, or
42 square inches. Division works in the corresponding inverse way.\chapter{enumerations}


Sometimes you want a collection of distinct named values, for example as
the names of options ({\bf small}, {\bf medium}, {\bf large}, {\bf jumbo}). Rather than using
{\bf const} integer variables (which can be confused with plain integer values) or
strings/symbols (vulnerable to silent mis-spellings), Spice allows you to
define {\em enumeration values}:

\begin{quote}
\begin{verbatim}
define enum Size = small, medium, large, jumbo enddefine
\end{verbatim}
\end{quote}
{\bf Size} becomes a new type name, and {\bf small}, {\bf medium}, {\bf large}, and {\bf jumbo}
become new values of that type. {\bf print(medium)} will print {\bf medium}; the
values retain their names. Each value also has an associated small integer,
starting at {\bf 1} (for the first) and taking on successive values; you can see
this number using {\bf magnitude}, so {\bf large.magnitude.println} will print {\bf 3},
and you can construct enum values from numbers: {\bf new Size(4)} is identical to
{\bf jumbo}. Using values outside the range will generate an error.

Spice's enumerations are just shorthand. An {\bf enum} type is a class
extending the built-in {\bf Enum} class, which has slots for the name and
magnitude of its values. Each {\bf enum} value is an instance of its class
with the slots set appropriately.

\begin{quote}Spice implementors are encouraged to implement {\bf enum} values as efficiently as
possible, taking advantage of their specialised nature.
\end{quote}\chapter{all sorts of unmentionable things}


\section{extended relational expressions}


In Spice, the relational operators have a special syntactic property;
you can write expressions like {\bf x $<$= y $<$= z} to mean {\bf (x $<$= y) and (y $<$= z)}.
The expression {\bf y} will be evaluated at most once. The operators do not have
to be the same ({\bf x == y $<$= z} is fine) and you can chain them up as long as
you like ({\bf x == y == z == h == q} is fine, too).\section{op}


Sometimes you'd like to get at the functions that implement Spice's infix
operators, usually so that you can pass them to higher-order procedures
(eg for zipping two arrays together with a specified procedure such as
addition). The form {\bf op Operator}, where {\bf Operator} is a Spice operator,
gets a procedure that has the effect of that operator; so {\bf op +} is a
procedure of two arguments that adds them together and {\bf op ==} is a
procedure of two arguments that tests them for structural equality.

Note that {\bf op and} and {\bf op or} deliver ordinary procedures of two arguments
(the special syntactic property that means that {\bf E$_ 1$ and E$_ 2$} need not
evaluate {\bf E$_ 2$} is lost) as does {\bf op ==}, {\bf op $<$=}, etc (the chaining property
is lost).\section{super and extends}


Sometimes, in an overloaded procedure definition, you wish to invoke
the more general version of that procedure; you do the general case, then you
do your more specific one. The identifier {\bf super} in a procedure is bound to
the next most general definition of that procedure.

An initialiser (ie a {\bf method new}) {\em must} invoke its more general case,
and what's more, must do so before it starts its own initialisation. Rather
than using {\bf super} (which in any case doesn't have quite the right effect),
that pre-initialisation is described in the {\bf extends} clause of the
initialiser, which provides calls to the other initialisers. If the extends
clause is left out, it is as though you'd written {\bf extends Parent()} where
{\bf Parent} is the parent class of this one.\section{shared slots}


A class can have a shared slot, introduced by {\bf shared Name}, typed if desired.
A shared slot is a single location accessible through any instance but shared
by them all.

\begin{quote}Note that a shared slot is different from a variable declared in a class;
it is accessed by method-call syntax, as ordinary slots are, rather than
simply by its name, as variables are.\end{quote}\section{parallel iteration}


Spice allows {\bf for} loops to iterate over several things "in parallel".
Suppose {\bf Xs} and {\bf Ys} are both arrays, and we wish to "zip them together",
say by adding corresponding elements; we can write

\begin{quote}
\begin{verbatim}
{for x in Xs also y in Ys do all x + y endfor}
\end{verbatim}
\end{quote}
to make the array of sums. The loop will finish when either of {\bf Xs} or {\bf Ys}
are exhausted. As well as {\bf Name in Expr}, you can use the form {\bf from E$_ low$ to
E$_ high$}, with the {\bf to} part being optional:

\begin{quote}
\begin{verbatim}
for x in E also y from 1 do
    println( "element", y, "is", x )
endfor
\end{verbatim}
\end{quote}
Note that this works even if {\bf E} is not an array, and even if it is, is better
than

\begin{quote}
\begin{verbatim}
for y from 1 to E.length do
    println( "element", y, "is", E[y] )
endfor
\end{verbatim}
\end{quote}
because in the first case the compiler can generate the most efficient code
for indexing through the array (and the reader does not have to worry that
the iteration is properly bounded).

As well as {\bf from} and {\bf in} parts, you can specify termination conditions for
the loop with {\bf while} and {\bf until} clauses; if a {\bf while} clause goes {\bf false}
(or an {\bf until} goes {\bf true}) the loop ends "abnormally" (as it does with a
{\bf break}) and you can specify a final loop value with {\bf then}.



\part{the Spice reference manual}
\chapter{introduction}


\section{a note on syntax}


Syntax is described using an extended BNF. Terminal symbols that are names are
in {\bf bold}; terminals that are symbols are in bold and quoted, eg, "{\bf ++}".
Non-terminals are in {\tt this} font.

Grouping is shown by round brackets. Optional elements are enclosed in square
brackets, eg in {\bf "?" {[}Integer{]}} the {\bf Integer} may be omitted. A suffix
operator {\bf *} is used to mean "zero or more of", while a postfix operator {\bf +}
means "one or more of".

The infix operator {\bf E ** X} is used to mean "Zero or more {\bf E}s separated by
{\bf X}s"; {\bf X} is usually a comma or semicolon. Similarly infix {\bf E ++ X} means
"{\em one} or {\bf E}'s separated by {\bf X}'s".

All the syntax definitions in the body of the document are collected together
in an appendix.\chapter{lexis}


A Spice program is made up of a series of {\em tokens}. Tokens are
separated by {\em token gaps}. Token gaps are unimportant except that they
may serve to separate tokens that would otherwise combine into single
tokens (eg the two tokens "x" and "then" must be separated by a gap,
otherwise they would be recognised as a single token "xthen").

A token gap is any sequence of spaces, newlines, horizontal or
vertical tabs, and form-feeds, possibly containing comments (see below).

A token is a {\em word}, a {\em mark}, or a {\em literal}. A word is a sequence of
letters and digits; a mark is a sequence of mark characters; and a
literal is a string, character, numeric, or unit constant. The lexis
makes no substantive difference between those words (or marks) that
are reserved and those that may freely be used as identifiers (or
operators).

\section{words}


A word is a series of digits and letters starting with a letter. Words are
case sensitive and of unlimited length.

Spice conventions are that type names start with upper-case letters
and that value names do not.\section{marks}


A mark is either a {\em simple mark} or a {\em compound mark}. A {\em simple mark}
is one of the characters "(", ")", "\{", "\}", "{[}", "{]}", ";", ",". A
{\em compound mark} is a series of mark characters, where a mark character
is one of "~!@\%$^\wedge$\&*-+=$\mid$:.?/$<$$>$". Note that "\#" is {\em not} a mark
character.

Note that this means that "$<$$>$", "+++", "$\mid$-$>$" etc are all possible Spice
marks; they are reserved for future expansion.\section{literals}


There are four kinds of literal in Spice: numeric literals, string literals,
character literals, and regular-expression literals. The
numeric literals may be decorated with units.

\subsection{numeric and unit literals}


A numeric literal represents an integer or floating-point number, possibly
with units, possibly in a radix other than the default {\bf 10}. Within a numeric
literal, underbar characters may be present; they are immediately discarded,
being there only for presentation (eg to make {\bf 7\_476\_294} readable).

\subsubsection{plain integer literals}


A plain integer literal is a series of decimal digits. It represents the
obvious number in radix {\bf 10}. There is no limit to the number of digits
permitted, and the correct integer value is preserved, using bignums if
necessary.\subsubsection{radix-specified integer literals}


A radix-specified integer literal consists of a radix specifier, which is a
series of decimal digits followed by the letter {\bf x} (in either case), followed
by a series of letters and digits. The letters must be consistent with the
radix, in that if the radix is {\bf k} then no letter past {\bf a + (k - 10)} is
permitted; if the radix is {\bf 10} or less, any letters terminate the numeric
part of the literal.\subsubsection{integer literals with units}


A plain integer literal, or a radix-specified integer literal with radix {\bf 10}
or less, may be immediately followed by a unit name, which is a series of
letters. The integer value represents a value in those units. Thus {\bf 45cm},
{\bf 19mile}, and {\bf 6seconds} are all legal, assuming that those units have been
declared.

The value denoted by {\bf nU}, where {\bf n} is the number and {\bf U} the units, is the
measure value with {\bf n} as its magnitude and the unit named by {\bf U} as its unit.\subsubsection{plain floating literals}


A floating literal is a series of decimal digits, a decimal point, and another
series of decimal digits; it represents the "obvious" floating-point number.
It may be followed by an exponent specifier, which is the letter {\bf e},
optionally followed by a sign ({\bf +} or {\bf -}), followed by a plain integer
literal.\subsubsection{radix-specified floating literals}


A radix-specified floating literal is a radix specifier followed by a series
of letters and digits, a decimal point, a series of letters and digits, and an
optional exponent specifier.

The radix specifier gives the radix of the floating point number but not the
exponent. If the radix is {\bf 10} or less, then letters terminate the numeric
part of the literal. If the radix is more than {\bf 10}, the sign in any exponent
specifier is not optional.\subsubsection{floating literals with units}


A floating literal with radix {\bf 10} or less may be followed by a unit name, as
for integer literals with units.\subsubsection{general specification of units}


If a number in a radix greater than {\bf 10} is supposed to be in some units,
then one of the general syntaxes for values-in-units must be used.\subsection{string literals}


A string literal is a character sequence starting and ending with the
double-quote character {\bf "}. The characters between the quotes are any
characters except quotes (any flavour), newlines, control characters, or
backslashes except as permitted by {\em escape sequences}:

\begin{itemize}\item $\setminus$n -- a newline

\item $\setminus$f -- a form feed

\item $\setminus$v -- a vertical tab

\item $\setminus$r -- a return

\item $\setminus$b -- a backspace

\item $\setminus$t -- a horizontal tab

\item $\setminus$s -- a space

\item $\setminus$$\setminus$ -- a backslash

\item $\setminus$" -- a double quote

\item $\setminus$' -- a single quote

\item $\setminus$` -- a reverse quote

\item $\setminus$"(" numeric literal * "," ")")
-- the series of characters with those
Unicode values

\item $\setminus$u Hex Hex Hex Hex -- the Unicode character with value given by the
specified four hexadeciaml digits.

\end{itemize}Any other backslash sequence {\bf $\setminus$X} means "protected {\bf X}"; protection applies
within regular expression literals, {\em qv}.\subsection{character literals}


A character literal has the same syntax as a string literal, except that the
delimiters are the single quote {\bf '} instead of the double quote.

A character literal denotes the multiple values formed by all its constituent
characters. Thus {\bf ''} denotes no values, {\bf 'x'} denotes the single character
{\bf x}, and {\bf 'spoo'} denotes the characters {\bf s}, {\bf p}, {\bf o}, {\bf o}.\subsection{regular expression literals}


Spice has two regular expression syntaxes: a {\em traditional} syntax and a
{\em native} syntax. The traditional syntax is based on existing
regular-expression notations with little modification, and the native
syntax is designed to mesh cleanly with the string syntax. Regular expressions
are a late addition to Spice, and many details have not been settled, pending
experimental implementation.

A traditional regular expression literal starts with the sequence {\bf //} and
then is a sequence of characters terminated by a closing {\bf /}. The sequence
cannot include any unprotected {\bf /}. The exact permitted syntax of the sequence
is under discussion, but we expect to conform as far as is possible with the
syntax used by the Java regex package.

A native regular expression (NRE) starts with the sequence {\bf /"} and then is a
sequence of characters terminated by a closing {\bf "}; the sequence cannot
include an unprotected {\bf "}. Within a native regular expression, all the usual
string escapes apply with the same meaning, and protected characters stand for
themselves and have no meta-meaning.

Our proposal is that the characters {\bf ?$\mid$*\{\}{[}{]}\%} are special {[}that is, do not
stand for themselves{]} within NREs. {\bf \%} is the {\em marker character} and the rest
are meta-characters.

\begin{itemize}\item {\em character classes}, {[}XYZ{]}: the square brackets enclose compact
character-class expressions.

\item {\em grouping}, \{E\}: braces are grouping brackets which have no effect on
back-references.

\item {\em repetition}, E*: suffix * is the Kleene star.

\item {\em wildcard}, ?: matches any character.

\item {\em alternation}, E$\mid$F: matches either E or F.

\end{itemize}Other syntactic jobs are performed by sequences introduced by the marker
character. They are expected to be less frequent and so a larger overhead
to use them is acceptable.\section{comments}


A comment is either the character "\#"
followed by all characters up to a newline or
end-of file, or the mark "/*" followed by any series of characters excluding
"*/" and then by "*/", or the mark ";;;" followed by all characters up to a
newline or end-of-file.\chapter{top-level syntax}


A Spice program is composed of a collection of {\em packages}. A package
consists of a {\em header}, which identifies the package and what it relies
on, and its {\em body}, which is a series of {\em definitions} possibly
interspersed with {\em executions}. Definitions define variables,
procedures, classes, and units; executions are expressions
that perform some run-time activity.

\section{programs}


\begin{tabular}{l}
{\bf def75. Program ::= }\\ 
\hspace*{3mm}{\tt Spice (Package* $\mid$ PackageBody)} \\ 
\end{tabular}

\begin{tabular}{l}
{\bf def76. Spice ::= }\\ 
\hspace*{3mm}{\tt "spice" String ("," (Name ":" Expr))*} \\ 
\end{tabular}

The {\bf String} identifies the version of Spice that the writer assumed
when the package was written; it is a dot-separated sequences of
integers or names (eg "1.0" or "5.beta"). The {\bf Name}-{\bf Expr} pairs
define settings for compiler preferences; each {\bf Expr} must be
evaluable at compile-time.

The {\bf spice} preamble is intended for future expansion.\section{aside: modifiers}


{\bf Modifiers} are annotations to declarations. They allow a declaration to be
accompanied by arbitrary information. Modifiers can appear in two different
positions, {\em open} modifiers and {\em closed} modifiers; open modifiers can
decorate declarations not introduced by {\bf define} (ie, {\bf var}, {\bf const}, and
{\bf slot}), and thus need a more restricted syntax.

\begin{tabular}{l}
{\bf def77. OpenModifier ::= }\\ 
\hspace*{3mm}{\tt MarkedModifier Modifier*} \\ 
\end{tabular}

\begin{tabular}{l}
{\bf def78. Modifier ::= }\\ 
\hspace*{3mm}{\tt MarkedModifier} \\ 
\hspace*{3mm}{\tt  $\mid$ Name {[}(ModArgument){]}} \\ 
\hspace*{3mm}{\tt  $\mid$ "{[}" Modifier** "," "{]}"} \\ 
\end{tabular}

\begin{tabular}{l}
{\bf def79. MarkedModifier ::= }\\ 
\hspace*{3mm}{\tt "public"} \\ 
\hspace*{3mm}{\tt  $\mid$ "private"} \\ 
\hspace*{3mm}{\tt  $\mid$ "protected"} \\ 
\end{tabular}

\begin{tabular}{l}
{\bf def80. MarkedArgument ::= }\\ 
\hspace*{3mm}{\tt Literal} \\ 
\hspace*{3mm}{\tt  $\mid$ "(" Expression ")"} \\ 
\end{tabular}

Modifiers are
discussed in the sections that require them, except we shall mention
here the two marked modifiers:

\begin{itemize}\item {\bf public}: the identifier(s) are put into the public interface of this
package, ie, they can be used in packages that import this one.

\item {\bf private}: the identifier(s) are {\em not} put into the public interface
of this package; they cannot be referred to outside it at all.
{\bf private} is the default, except for identifiers declared in {\bf public}
packages, as below.

\item {\bf protected}: protected identifiers are public but can only be referred to in
other packages using the {\bf alias::name} notation; see also {\bf protected import}.

\end{itemize}\section{packages}


\begin{tabular}{l}
{\bf def81. Package ::= }\\ 
\hspace*{3mm}{\tt \begin{tabular}{l}
MarkedModifier \\ 
 "package" \\ 
 PackageName \\ 
 PackageBody \\ 
 {[}("endpackage"){]} \\ 
\end{tabular}
} \\ 
\end{tabular}

\begin{tabular}{l}
{\bf def82. PackageName ::= }\\ 
\hspace*{3mm}{\tt Word++ "."} \\ 
\end{tabular}

A package name is a sequence of dot-separated identifiers. The last name
in the sequence is the {\em leafname} of the package; apart from in
{\bf package} and {\bf import} constructs, packages are referred to only by
their leafname or an alias for it.

Top-level identifiers declared in a {\bf public package} are public by default.
Top-level identifiers declared in a {\bf private package} are private by default.

\begin{tabular}{l}
{\bf def83. PackageBody ::= }\\ 
\hspace*{3mm}{\tt Import** SEMI Bundle** SEMI} \\ 
\end{tabular}

\begin{tabular}{l}
{\bf def84. Bundle ::= }\\ 
\hspace*{3mm}{\tt Definition+ Expr*} \\ 
\end{tabular}

A package body consists of its import declarations followed by its own
definitions. All the names exported by the packages which have been imported
are available in all the bundles of the package. Each bundle is a sequence
of definitions, possibly followed by some expressions; each {\bf Bundle} is as
long as possible. The names declared by a {\bf Bundle} are {\em visible throughout
that bundle}; declarations need not lexically preceed their use.

Expressions are evaluated when the package is loaded.

\begin{tabular}{l}
{\bf def85. Definition ::= }\\ 
\hspace*{3mm}{\tt ProcedureDef} \\ 
\hspace*{3mm}{\tt  $\mid$ VarDef} \\ 
\hspace*{3mm}{\tt  $\mid$ ClassDef} \\ 
\hspace*{3mm}{\tt  $\mid$ UnitDef} \\ 
\hspace*{3mm}{\tt  $\mid$ EnumDef} \\ 
\end{tabular}

A definition is a procedure, class, unit, enumeration, or
variable definition.\section{imports}


\begin{tabular}{l}
{\bf def86. Import ::= }\\ 
\hspace*{3mm}{\tt \begin{tabular}{l}
MarkedModifier \\ 
 "import" \\ 
 OpenModifier \\ 
 {[}(Alias "="){]} \\ 
 PackageName \\ 
 {[}("facet" (Name** ",")){]} \\ 
 {[}("from" Expr){]} \\ 
\end{tabular}
} \\ 
\end{tabular}

An {\bf import} definition identifies a package to be imported and possibly
a location to import it from; that location is given as a
String-valued expression representing a URL. The name can be given
directly or by the evaluation of an expression.

If the modifier {\bf public} is given, then all the identifiers in the
imported package are added to this package's public interface. This
allows packages to act as "collectors" for definitions from several
other places.

An imported identifier {\bf x} can be referred to as {\bf leafname::x}, where
{\bf leafname} is the leafname of the package (if {\bf Alias=} is omitted) or
the {\em alias} {\bf name} (if {\bf Alias=} is given). Unless the modifier {\bf protected} is
given, it may also be referred to as {\bf x}.

If two (or more) packages with the same leafname are imported, at least one
of them must be given an alias. If two (or more) packages export the same
identifier {\bf x}, then {\bf x} must be referred to using the {\bf alias::x} notation.

If the {\bf facet} modifier is present, only the names in the specified facets
are imported (see elsewhere for facet definitions).\section{procedure definitions}


The full power of a procedure definition is available through {\bf define}
options, which allow generics, methods, plain functions, and initialisers
to be declared.

\begin{tabular}{l}
{\bf def87. ProcedureDef ::= }\\ 
\hspace*{3mm}{\tt FullProcedureDef} \\ 
\hspace*{3mm}{\tt  $\mid$ FunctionDef} \\ 
\hspace*{3mm}{\tt  $\mid$ CompactDef} \\ 
\end{tabular}

\begin{tabular}{l}
{\bf def88. FullProcedureDef ::= }\\ 
\hspace*{3mm}{\tt \begin{tabular}{l}
"define" \\ 
 Modifier \\ 
 ("method" $\mid$ "function" $\mid$ "generic" $\mid$ "init" $\mid$ "specific") \\ 
 Header \\ 
 {[}("returns" Type){]} \\ 
 {[}("extends" CommaExpr){]} \\ 
 {[}("super" Header){]} \\ 
 ProcedureBody \\ 
 "enddefine" \\ 
\end{tabular}
} \\ 
\end{tabular}

\begin{tabular}{l}
{\bf def89. FunctionDef ::= }\\ 
\hspace*{3mm}{\tt \begin{tabular}{l}
OpenModifier \\ 
 "function" \\ 
 Header \\ 
 {[}("returns" Type){]} \\ 
 {[}("extends" CommaExpr){]} \\ 
 {[}("super" Header){]} \\ 
 ProcedureBody \\ 
 "endfunction" \\ 
\end{tabular}
} \\ 
\end{tabular}

\begin{tabular}{l}
{\bf def90. CompactDef ::= }\\ 
\hspace*{3mm}{\tt "def" Header ("=" $\mid$ "as" $\mid$ "=$>$") Expr} \\ 
\end{tabular}

A procedure declared by {\bf function} or {\bf def} is monomorphic; it has a single
definition and cannot be redeclared.

A procedure declared by {\bf generic} is a polymorphic procedure; the {\bf generic}
definition gives its most general argument types and its default body.

A procedure declared by {\bf specific} is a dynamic overloading of the named
generic procedure (which must exist and be visible).

If a {\bf generic}, {\bf specific}, or {\bf method} definition appears within a class, it
is given an implicit first argument called {\bf this} whose type is that class,
unless the {\bf plain} modifier has been supplied. Outside a class, {\bf plain} does
nothing. {\bf function} and {\bf def} never supply a {\bf this}.

A procedure declared with {\bf init} is an {\em initialiser}: it is used to fill in
the fields of freshly-constructed objects.

{\bf init} procedures may only be declared within classes. A name used
for a {\bf init} procedure cannot be used as the name of an ordinary
procedure; also they can be defined only with one class.

The {\bf Header} defines the procedure argument's names and types, the procedure
name itself, and the "expected shape" of calls to the procedure. {\bf returns}
gives the type of the value(s) returned by a procedure; if omitted, a return
type is inferred from the body of the procedure.

{\bf extends} and {\bf super} are currently under revision.

\begin{tabular}{l}
{\bf def91. Header ::= }\\ 
\hspace*{3mm}{\tt CallShape} \\ 
\hspace*{3mm}{\tt  $\mid$ Name "-$>$" CallShape} \\ 
\hspace*{3mm}{\tt  $\mid$ CallShape ":=" Arg} \\ 
\end{tabular}

The allowed {\bf Modifier}s include {\bf public}, {\bf protected}, and {\bf private}. A
procedure with modifier {\bf frozen} cannot be overloaded outside its package; a
procedure with modifier {\bf fluid} can be. (By default, {\bf function}s are frozen
and {\bf method}s are not.) An overloading of a procedure which appears outside
its owning package must have the modifier {\bf extension}, unless it appears in a
{\bf class} and extends a method of that {\bf class}.

If the updater argument is given, this definition defines the {\em updater} of the
named procedure, which must be declared in this package and have a compatible
{\bf CallShape}.

\begin{tabular}{l}
{\bf def92. CallShape ::= }\\ 
\hspace*{3mm}{\tt PrefixShape} \\ 
\hspace*{3mm}{\tt  $\mid$ DottedShape} \\ 
\hspace*{3mm}{\tt  $\mid$ InfixShape} \\ 
\end{tabular}

\begin{tabular}{l}
{\bf def93. PrefixShape ::= }\\ 
\hspace*{3mm}{\tt NameA {[}(Arglist){]}} \\ 
\end{tabular}

\begin{tabular}{l}
{\bf def94. DottedShape ::= }\\ 
\hspace*{3mm}{\tt NameB . NameA {[}(Arglist){]}} \\ 
\end{tabular}

\begin{tabular}{l}
{\bf def95. InfixShape ::= }\\ 
\hspace*{3mm}{\tt ArgA @ NameA {[}(ArgB){]}} \\ 
\end{tabular}

There are three different {\bf CallShape}s; the conventional prefix form
{\bf f(x)}, the method-oriented form {\bf x.f(a)}, and the infix-oriented form
{\bf x @f y}. {\bf NameA} is the name of the procedure being defined in all
cases. The arguments are given by the optional {\bf Arglist} for the
prefix form, by the {\bf Arglist} and {\bf NameB} for the method form, and by
{\bf ArgA} and {\bf ArgB} for the infix form; {\bf ArgA} and {\bf ArgB} are both
{\bf Arg}s, see below.

A {\bf method} is implicitly in method form, with {\bf this.} assumed as the first
argument unless overridden explicitly. For {\bf init} it is permitted
to omit the {\bf NameA}; in this case, the initialiser is that called by
expressions of the form {\bf new C}.

\begin{tabular}{l}
{\bf def96. Arglist ::= }\\ 
\hspace*{3mm}{\tt "(" Args ")"} \\ 
\end{tabular}

\begin{tabular}{l}
{\bf def97. Args ::= }\\ 
\hspace*{3mm}{\tt Arg** ","} \\ 
\end{tabular}

\begin{tabular}{l}
{\bf def98. Arg ::= }\\ 
\hspace*{3mm}{\tt Name {[}(":" {[}("!"){]} Type $\mid$ "==" Expr){]}} \\ 
\hspace*{3mm}{\tt  $\mid$ {[}(Name){]} "..."} \\ 
\end{tabular}

An {\bf Arglist} is just a comma-separated list of {\bf Arg}s. An {\bf Arg} is a {\bf Name},
optionally followed by a {\bf Type} introduced by a {\bf TypeWord}, or optionally
followed by a value introduced by "=="; or by {\bf ...}, making it an {\em indefinite}
argument, bound to an array of all the extra arguments to the procedure.

At most one indefinite argument is allowed, but it may appear anywhere in the
argument list.

\begin{quote}All argument identifiers are immutable, ie, cannot be assigned to, but
if they are data-structures, their components can be changed.
\end{quote}The prefixing of the {\bf Type} with the symbol {\bf !} means that the argument
does not take part in overloaded dispatch; all overloadings of that procedure
must have that argument, and mark it as non-dispatch.

{\bf == Expr} is equivalent to {\bf : T} where {\bf T} is a type whose only value is the
value of {\bf Expr}, which is evaluated at compile-time.

\begin{tabular}{l}
{\bf def99. ProcedureBody ::= }\\ 
\hspace*{3mm}{\tt StatementSeq} \\ 
\hspace*{3mm}{\tt  $\mid$ ArgList "=$>$" ProcedureBody} \\ 
\hspace*{3mm}{\tt  $\mid$ Arg "=$>$" ProcedureBody} \\ 
\end{tabular}

A procedure body is introduced by {\bf =$>$} or {\bf as} and ends with {\bf enddefine}. If
the body is empty, the arrow may be omitted. If the body has the form {\bf A=$>$B},
the procedure is {\em curried}; the body is itself a function expression.

\subsection{multiple definition}


A {\bf generic} procedure may have many different definitions, with the rest being
given as {\bf method}s. The different definitions may have different numbers of
arguments (up to any indefinite arguments) and no two definitions can have the
same sequence of types for their arguments.

Furthermore, for any pair of definitions with signatures (ie the
type-sequence of its arguments) {\bf T$_ 1$} and {\bf T$_ 2$}, there must be a
definition with type-sequence {\bf T$_ 3$} where the {\bf i}th element of {\bf T$_ 3$}
is a super-type of the {\bf i}th elements of {\bf T$_ 1$} and {\bf T$_ 2$} (note that
{\bf T$_ 3$} might be equal to one of {\bf T$_ 1$} or {\bf T$_ 2$}).

A call of such a polymorphic method will execute the most specific instance of
it, ie, the one whose formal arguments are as specific as possible while
matching the actual arguments. A procedure with {\bf K} arguments does not match a
call with {\bf N /== K} arguments, unless it has a final indefinite argument and
{\bf N$>$K}.

In particular, methods can be redeclared in classes, when
they over-ride the definitions in their parents.

\begin{quote}If a method has definitions of different arities, all definitions with
indefinite arguments are, for the purposes of dispatch, extended to the same
length by adding additional penultimate arguments of type {\bf Any}.\end{quote}\subsection{super and extends}


An overloaded procedure may wish to invoke the "next most general" definition
to complete (or start) its work. The identifier {\bf super} is bound to that
next most general definition. When there is ambiguity about that, it is
resolved by the {\bf super} clause of the function, which takes the form of a
function call where the operands are all type names; {\bf super} is bound to the
version of the function that takes those types as its argument.

\begin{quote}This use of {\bf super} has not been tested in the fire of the
    current Spice implementation.
\end{quote}An initialiser in a class {\bf C} with parent class(es) {\bf A (B...)} is
required to invoke initialisers of all of those classes. This is done in
the {\bf extends} clause of the initialiser; this clause is a comma-separated list
of initialiser calls, where each called procedure must be an initialiser
of a parent class and each parent class must have at least one initialiser
called.

\begin{quote}Similarly, this use of {\bf extends} has not been tried in practice.
\end{quote}\section{variable definitions}


A variable declaration introduces one or more identifiers, may give
their initial value, and may give their type.

\begin{tabular}{l}
{\bf def100. VarDef ::= }\\ 
\hspace*{3mm}{\tt MarkedModifier ("const" $\mid$ "var") NameDecl} \\ 
\end{tabular}

\begin{tabular}{l}
{\bf def101. NameDecl ::= }\\ 
\hspace*{3mm}{\tt \begin{tabular}{l}
(OneDecl $\mid$ "(" OneDecl++ "," ")") \\ 
 {[}(("=" $\mid$ ":=") Expr){]} \\ 
\end{tabular}
} \\ 
\end{tabular}

\begin{tabular}{l}
{\bf def102. OneDecl ::= }\\ 
\hspace*{3mm}{\tt Name {[}(":" Type $\mid$ "..."){]}} \\ 
\end{tabular}

The {\bf Modifier} may include {\bf public}, {\bf protected}, and {\bf private}. A {\bf var} may
have modifier {\bf readonly}, in which case it cannot be updated outside of its
defining package; {\bf const}s can never be assigned to. All the names declared in
a single {\bf const} or {\bf var} declaration share the same {\bf Modifier}.

The {\bf NameDecl} may initialise one or more names to the value(s) of an
{\bf Expr}; if the {\bf Expr} is omitted, they are initialised to the default value of
their type; if no type is specified, they are given type {\bf Any}.

A {\bf NameDecl} will usually contain a single {\bf OneDecl}, but it may contain
several, in which case the {\bf Expr} must be present and they are initialised
from it as for assignment. The {\bf ...} form is permitted for at most one
{\bf OneDecl}; it bundles all the (remaining) values from the {\bf Expr}
into an array which is assigned to the {\bf Name}.

It is forbidden for {\bf OneDecl}s to be typed if the entire {\bf NameDecl} is typed,
and {\em vice versa}.

When a {\bf var} or {\bf const} appears at top-level, or within a class,
it defines a permanent variable that comes into existance when the
package is loaded and goes away only when no references to it remain.

When a {\bf var} or {\bf const} appears within a procedure, it defines a local
variable; a new location is allocated each time the declaration is executed.
This location does not share with any other location. It persists as long
as there are any references to it, which means that if it is not captured
by any lambda-expression or hole-expression referring to it it may
be disposed of when its scope is left (if not sooner).\section{class definitions}


A Class defines a shape shared by a collection of objects which are its
instances. Every Class is itself an object, and has a particular instance
called its {\em prototypical instance}.

\begin{tabular}{l}
{\bf def103. ClassDef ::= }\\ 
\hspace*{3mm}{\tt \begin{tabular}{l}
"define" \\ 
 Modifier \\ 
 "class" \\ 
 Name \\ 
 {[}("extends" CommaExpr){]} \\ 
 ClassProperty** \\ 
 SEMI \\ 
 "enddefine" \\ 
\end{tabular}
} \\ 
\end{tabular}

The {\bf Name} is defined to be a constant bound to the class object,
and is suitable for use as a type name.

A class may be qualified as {\bf public}, {\bf protected} {\bf private},
{\bf array}, and {\bf abstract}. If
it is qualified {\bf array}, its instances are array-type objects, and can
be indexed with integers. A {\bf public} class has its {\bf method}s and
{\bf function}s (but not its {\bf slot}s) declared {\bf public} by default.
A class qualified {\bf abstract} cannot have direct instances; trying to {\bf new} it
will fail.

\begin{quote}The {\bf array} modifier is not implemented by the current Spice compiler.\end{quote}If the {\bf extends} clause is present, its {\bf CommaExpr} must be a comma-separated
sequence of names of other classes, and this class inherits all of
their slots. (Slots that are inherited twice along different routes of course
only count once in the new class.)

\begin{quote}The current Spice compiler only implements single inheritance.\end{quote}\begin{tabular}{l}
{\bf def104. ClassProperty ::= }\\ 
\hspace*{3mm}{\tt Procedure} \\ 
\hspace*{3mm}{\tt  $\mid$ SlotDecl} \\ 
\hspace*{3mm}{\tt  $\mid$ VarDef} \\ 
\end{tabular}

The properties of a class are its procedures and slots. A {\bf NameDecl} within a
class acts exactly like a {\bf NameDecl} outside a class -- they declare top-level
permanent identifiers. Within a class, method definitions by default give
their implicit {\bf this} argument that class as their type.

\begin{tabular}{l}
{\bf def105. SlotDecl ::= }\\ 
\hspace*{3mm}{\tt \begin{tabular}{l}
("slot" $\mid$ "shared") \\ 
 Name \\ 
 {[}(":" Type){]} \\ 
 {[}("=" Expr){]} \\ 
 {[}("implements" Name2){]} \\ 
\end{tabular}
} \\ 
\end{tabular}

Slots are declared by {\bf slot} and {\bf shared} declarations. Each {\bf slot}
declaration arranges that objects of this type have a slot in them, suitable
for storing values of the given {\bf Type}, and defines a procedure {\bf Name1} of one
argument which has the class as its type and whose action is to deliver the
value of its slot; that procedure also has an updater which alters the value
of that slot.

If {\bf implements Name2} is specified, then {\bf Name2} must be a generic procedure,
and the slot declaration provides an overloading of that procedure for
arguments of the current class.

A {\bf shared} declaration does not allocate any slots in the
object, but makes a procedure which accesses (and updates) a single shared
location.

If a slot or shared identifier is qualified {\bf readonly}, then the procedure
that it exports has no updater, although the locally-visible version does.

When an object of this type is constructed, the {\bf Expr}s of its {\bf slot}s are run
to get the initial values for those slots.

\begin{quote}{\bf shared} and {\bf readonly} are not implemented by the current Spice compiler.
\end{quote}A class definition for {\bf Spoo} allows the expression {\bf E is Spoo} as a test for
the value {\bf E} being an instance of {\bf Spoo} (or a subtype).\section{unit definitions}


\begin{quote}Unit definitions are not implemented by the current Spice compiler.\end{quote}A {\bf unit} definition informs the compiler about units and their dimensions.

\begin{tabular}{l}
{\bf def106. UnitDef ::= }\\ 
\hspace*{3mm}{\tt \begin{tabular}{l}
"define" \\ 
 ("unit" $\mid$ "units") \\ 
 UnitDef++ \\ 
 "," \\ 
 "enddefine" \\ 
\end{tabular}
} \\ 
\end{tabular}

\begin{tabular}{l}
{\bf def107. UnitDef ::= }\\ 
\hspace*{3mm}{\tt Word {[}("=" Expr){]}} \\ 
\end{tabular}

The {\bf Word} of a {\bf UnitDef} is declared to be legitimate unit names.
If the {\bf Expr} of a {\bf UnitDef} is present, it must be an compile-time expression
delivering an anonymous {\bf Unit} value, which is named by the {\bf Word}.

\begin{quote}for example, we may have {\bf define unit mile = inch * 12 * 3 * 1760
enddefine}, which defines {\bf mile} in terms of {\bf inch}, or {\bf define unit length =
new Unit(1,`length`) enddefine}, which defines a new unit of dimension
`length`.\end{quote}\section{enumeration definitions}


\begin{tabular}{l}
{\bf def108. EnumDef ::= }\\ 
\hspace*{3mm}{\tt \begin{tabular}{l}
"define" \\ 
 "enum" \\ 
 {[}(Modifier){]} \\ 
 Name \\ 
 "=" \\ 
 Name++ \\ 
 "," \\ 
 "enddefine" \\ 
\end{tabular}
} \\ 
\end{tabular}

An {\bf enum} declaration for {\bf Foo} is shorthand for declaring a new class called
{\bf Foo} extending {\bf Enum}, which provides two private slots (for the name and the
magnitude of enumeration values) and an instance of this class for each of the
{\bf Name}s. The enumeration values atart at {\bf 1} and the default value for {\bf Foo}
is an instance with value {\bf 0} and name {\bf nullFoo}.

Implementors are encouraged to find efficient representations for enumeration
values.\chapter{small syntax}


Most of the rest of the syntax of Spice is in terms of {\em expressions}.
Spice expressions may denote tuples of values; we say thay are
{\em multiple valued}. Spice expressions may be evaluated in three modes;
{\em value} mode, when they are evaluated for their value; {\em update} mode,
when they are evaluated to update some location or data structure; and
{\em void} mode, which is value mode except that all the values that are
delivered are then discarded.

Unless otherwise specified, an expression is illegal in update mode.
All function calls and operator applications are legal there; every
procedure {\bf P} can have an updater {\bf U}, which is the procedure that is
called when a call of {\bf P} appears as the target of an assignment.

\section{expressions}


A {\bf Statement} is an expression, or sequence of expressions separated
by commas. The value of a sequence of {\bf Statement}s is the value of the
{\em last} {\bf Statement}; the values of all the others are discarded.

\begin{tabular}{l}
{\bf def109. Statement ::= }\\ 
\hspace*{3mm}{\tt Expr++ "," SEMI} \\ 
\end{tabular}

\begin{tabular}{l}
{\bf def110. StatementSeq ::= }\\ 
\hspace*{3mm}{\tt Statement*} \\ 
\end{tabular}

\begin{tabular}{l}
{\bf def111. Expr ::= }\\ 
\hspace*{3mm}{\tt Name} \\ 
\hspace*{3mm}{\tt  $\mid$ Hole} \\ 
\hspace*{3mm}{\tt  $\mid$ Literal} \\ 
\hspace*{3mm}{\tt  $\mid$ Expr "..."} \\ 
\hspace*{3mm}{\tt  $\mid$ PreOp Expr} \\ 
\hspace*{3mm}{\tt  $\mid$ Expr PostOp} \\ 
\hspace*{3mm}{\tt  $\mid$ Expr InOp Expr} \\ 
\hspace*{3mm}{\tt  $\mid$ Expr AssignOp Expr} \\ 
\hspace*{3mm}{\tt  $\mid$ "(" Expr ")"} \\ 
\hspace*{3mm}{\tt  $\mid$ Expr {[}("." DotExpr){]} {[}(OptExpr){]}} \\ 
\hspace*{3mm}{\tt  $\mid$ Expr {[}("@" DotExpr){]} {[}(OptExpr){]}} \\ 
\hspace*{3mm}{\tt  $\mid$ Expr "{[}" Expr "{]}"} \\ 
\hspace*{3mm}{\tt  $\mid$ "{[}" Expr "{]}"} \\ 
\hspace*{3mm}{\tt  $\mid$ LambdaExpr} \\ 
\hspace*{3mm}{\tt  $\mid$ "new" Name {[}("(" Expr ")"){]}} \\ 
\hspace*{3mm}{\tt  $\mid$ "none" OptExpr} \\ 
\hspace*{3mm}{\tt  $\mid$ "one" Expr} \\ 
\hspace*{3mm}{\tt  $\mid$ "once" Expr} \\ 
\hspace*{3mm}{\tt  $\mid$ "throw" Expr} \\ 
\hspace*{3mm}{\tt  $\mid$ "{[}" CommaExpr "{]}"} \\ 
\hspace*{3mm}{\tt  $\mid$ "\{" CommaExpr "\}"} \\ 
\hspace*{3mm}{\tt  $\mid$ QuoteExpr} \\ 
\hspace*{3mm}{\tt  $\mid$ TryCatchExpr} \\ 
\hspace*{3mm}{\tt  $\mid$ IfExpr} \\ 
\hspace*{3mm}{\tt  $\mid$ RepeatExpr} \\ 
\hspace*{3mm}{\tt  $\mid$ ForExpr} \\ 
\hspace*{3mm}{\tt  $\mid$ SwitchExpr} \\ 
\hspace*{3mm}{\tt  $\mid$ VarDef} \\ 
\hspace*{3mm}{\tt  $\mid$ ProcedureDef} \\ 
\end{tabular}

Operator expressions are disambiguated by the "usual" precedence
rules. Here, {\bf Expr} is an expression that cannot contain top-level
commas.

\begin{tabular}{l}
{\bf def112. OptExpr ::= }\\ 
\hspace*{3mm}{\tt {[}(Expr){]}} \\ 
\end{tabular}

\begin{tabular}{l}
{\bf def113. CommaExpr ::= }\\ 
\hspace*{3mm}{\tt Expr++ ","} \\ 
\end{tabular}

\begin{tabular}{l}
{\bf def114. OptCommaExpr ::= }\\ 
\hspace*{3mm}{\tt {[}(CommaExpr){]}} \\ 
\end{tabular}

\subsection{name}


A {\bf Name} is either a simple identifier, or an imported identifier
{\bf short::id}, where {\bf short} is the short name of a package and {\bf id} is
the name of the identifier within it.\subsection{hole}


A Hole is a "virtual operand" used when making partial applications.
Thus the expression {\bf ? + 1} is a function that adds 1 to things. The
expression {\bf ?2 - ?1} is reverse subtract; the arguments to a partial
application are numbered {\bf 1..N}, left-to-right.

\begin{tabular}{l}
{\bf def115. Hole ::= }\\ 
\hspace*{3mm}{\tt "?" {[}(Integer){]}} \\ 
\end{tabular}

\subsection{literal}


Literals may be symbols, strings, characters, numbers, or reserved
literals. Symbols, strings, numbers, and reserved literals are
themselves lexical items.

\begin{tabular}{l}
{\bf def116. Literal ::= }\\ 
\hspace*{3mm}{\tt StringLiteral} \\ 
\hspace*{3mm}{\tt  $\mid$ NumberLiteral} \\ 
\hspace*{3mm}{\tt  $\mid$ ReservedLiteral} \\ 
\hspace*{3mm}{\tt  $\mid$ CharacterLiteral} \\ 
\end{tabular}

\subsection{explosions}


The value of the expression {\bf E...} is the {\em explosion} of the value of {\bf E}, ie
all the components of all the public values of {\bf E}.

In update mode (ie as the target of an assignment) {\bf E} must be an
updatable expression; it is assigned an array constructed of all the
values on the source of the expression.\subsection{prefix operators}


The value of a prefix expression is found by applying the procedure
bound to the prefix operator to the value delivered by the operand,
{\em except} for the special operator {\bf once}, which arranges that its
operand is evaluated only once, "as soon as possible".

The prefix operator {\bf one} delivers the first of its arguments, and
{\bf absent} if it called with no arguments.\subsection{postfix operators}


The value of a postfix expression is found by applying the procedure
bound to the postfix operator to the value delivered by the operand.

\begin{tabular}{l}
{\bf def117. PostOp ::= }\\ 
\hspace*{3mm}{\tt LexicalPostfixOperator} \\ 
\hspace*{3mm}{\tt  $\mid$ "@" DotExpr} \\ 
\end{tabular}

{[}Note that {\bf @} can serve as a infix or postfix call marker.{]}\subsection{infix operators}


\begin{tabular}{l}
{\bf def118. InOp ::= }\\ 
\hspace*{3mm}{\tt LexicalInfixOperator} \\ 
\hspace*{3mm}{\tt  $\mid$ "@" DotExpr} \\ 
\end{tabular}

{\bf Note}. The logical operators {\bf and} and {\bf or}, and the relational operators
{\bf ==}, {\bf $<$=}, {\bf $>$=}, {\bf $<$}, {\bf $>$}, {\bf !=}, {\bf /==}, {\bf ==}, {\bf /==} are special in that
they all automatically {\bf one} their arguments.

Further, the relational operators are {\em continued relationals}; if {\bf R} and {\bf S}
are relational operators, the expression {\bf a R b S c} is shorthand for {\bf a R b
and b S c}, except that {\bf b} is only evaluated once.\subsection{assignment operators}


The assignment operators {\bf X:=} for some operators {\bf X} (ie {\bf +:=}, {\bf -:=},
{\bf *:=}, {\bf /:=}, {\bf \&:=}, {\bf $\mid$:=}, {\bf $^\wedge$:=}) are shorthand for assignments where
the target and left operand of {\bf X} are the same, ie, {\bf E$_ 1$ X:= E$_ 2$} is
equivalent to {\bf E$_ 1$ := E$_ 1$ X E$_ 2$} except that side-effects of {\bf E$_ 1$}
are only applied once.

The assignment operator {\bf :=} evaluates its left operand (the {\em target}) in
update mode with its right operand being the {\em source}.

The assignment operator "-$>$" evaluates its {\em right} operand (the target) in
update mode, with the source being its {\em left} operand; any excess values are
retained as the value of the assignment expression.

The source delivers some sequence of values; the target is a sequence of
updateable expressions. At most one target is allowed to be a procedure call
with an updater which takes an indefinite number of assigned values.
(Otherwise, in {\bf (f(), g()) = (1,2,3)}, where {\bf f} and {\bf g} both have indefinite
updaters, there would be no way to tell how the values were to be distributed
between {\bf f} and {\bf g}.) Assignments are done right-to-left; each
assignment target takes the appropriate number of values from the right-hand
end of the value sequence and the rest of the value sequence is handed to the
remaining targets. If the assignment target is an identifier, one value is
assigned into it. If it is a procedure call, its updater is invoked, passing
it the appropriate number of values.\subsection{bracketed expressions}


The expression {\bf (E)} has the same value that {\bf E} does.\subsection{call expressions}


The expression {\bf x.f} is a call of {\bf f} with argument(s) {\bf x}, as are the
expressions {\bf f(x)} and {\bf x @f}. The expression {\bf x.f(y)} is a call of
{\bf f} with arguments {\bf (x,y)}, as is {\bf x @f y}.

When a call appears as the target of an assignment, eg {\bf f(x)=E}, it is
equivalent to the expression {\bf (f.updater)(x)(E)}; that is, the updater
function specialises itself on the target arguments {[}allowing it to
be overloaded on those arguments{]} and delivers a function which consumes
the assigned values {[}and which may be separately overloaded on those values{]}.

\begin{quote}The current Spice implementation uses an earlier definition that has
a similar effect in simple cases, but is inadequate to deal with
overloaded functions: {\bf (f.updater)(x,E)}.\end{quote}\begin{tabular}{l}
{\bf def119. DotExpr ::= }\\ 
\hspace*{3mm}{\tt Name} \\ 
\hspace*{3mm}{\tt  $\mid$ "(" Expr ")"} \\ 
\hspace*{3mm}{\tt  $\mid$ "return"} \\ 
\hspace*{3mm}{\tt  $\mid$ "throw"} \\ 
\hspace*{3mm}{\tt  $\mid$ "new"} \\ 
\end{tabular}

The expression permitted after a dot or {\bf @} may be a {\bf Name}, an arbitrary
expression in brackets, or one of the reserved words {\bf throw}, or
{\bf return} (which allows these forms to be used postfix) or {\bf new} (which
allows it to be used infix).\subsection{array expressions}


The expression {\bf \{E\}} is an {\em array expression}; it creates a new array
by evaluating its operand and piling all the results into the new
array.\subsection{index expressions}


The value of the expression {\bf a{[}i{]}} is that obtained by calling the
{\em index function} associated with the type of {\bf a}, supplying it with
{\bf a} and {\bf i}. If {\bf a} is an array, this attempts to index the array. If
{\bf a} is a table, it looks {\bf i} up in that table. Otherwise it attempts
to find the {\bf i}-property of {\bf a}.

When an index expression is used in update mode, the updater of the
associated index function is called.\subsection{lambda expressions}


A lambda-expression is an inline procedure definition; it is the
procedure taking the specified arguments and computing the desired
result.

\begin{tabular}{l}
{\bf def120. LambdaExpr ::= }\\ 
\hspace*{3mm}{\tt "(" Args "=$>$" StatementSeq ")"} \\ 
\hspace*{3mm}{\tt  $\mid$ "fun" Args "=$>$" StatementSeq "endfun"} \\ 
\end{tabular}

\subsection{quotation expressions}


A quotation expression allows {\bf Symbol} values, and sequences of {\bf Symbol}
values, to be denoted by an expression.

\begin{tabular}{l}
{\bf def121. QuoteExpr ::= }\\ 
\hspace*{3mm}{\tt "`" QuotedItems "`"} \\ 
\end{tabular}

\begin{tabular}{l}
{\bf def122. QuotedItems ::= }\\ 
\hspace*{3mm}{\tt QuotedItem*} \\ 
\end{tabular}

\begin{tabular}{l}
{\bf def123. QuotedItem ::= }\\ 
\hspace*{3mm}{\tt Word} \\ 
\hspace*{3mm}{\tt  $\mid$ "\{" QuotedItems "\}"} \\ 
\hspace*{3mm}{\tt  $\mid$ "(" Expr ")"} \\ 
\hspace*{3mm}{\tt  $\mid$ "$^\wedge$" Word} \\ 
\hspace*{3mm}{\tt  $\mid$ "$^\wedge$$^\wedge$" Word} \\ 
\hspace*{3mm}{\tt  $\mid$ Literal} \\ 
\end{tabular}

A {\bf QuotedItem} that is a literal or a word denotes the corresponding value;
words are represented as Spice {\bf Symbol}s (interned strings with a type of
their own). An item which is an expression in brackets evaluates to the
value of that expression; the form {\bf $^\wedge$Word} is equivalent to {\bf (Word)},
with {\bf Word} stripped of any special syntactic properties, and the form
{\bf $^\wedge$$^\wedge$Word} is equivalent to {\bf (Word...)}.

The form {\bf \{Q\}} denotes the sequence formed out of the values denoted by {\bf Q}.
Thus quotations can express nested sequence structures which are mostly
constant.\subsection{new expressions}


The expresssion {\bf new X(A)} is a construction expression; {\bf X} is a constructor
value, and {\bf A} the arguments to the construction. If the argument {\bf E} is
omitted, it does {\em not} mean that {\bf X} is invoked on no values; instead
the constructor {\em function} is delivered, as though one had written

\begin{quote}
\begin{verbatim}
(A... => new X(A...))
\end{verbatim}
\end{quote}
If {\bf X} is a class value, then a new default-initialised instance of that class
is constructed, and handed to the class's anonymous constructor. If {\bf X} is a
named constructor, that is, defined as {\bf new X} in some class {\bf C}, a new empty
instance of {\bf C} is constructed and passed to {\bf X} with arguments {\bf A}.\subsection{none expressions}


The expression {\bf none} denotes no values. The expression {\bf none E}
evaluates {\bf E} and discards all the values it computes.\subsection{throw and try expressions}


The expression {\bf throw E} raises an exception with value the value of {\bf E}. If
this exception is not caught within the program, it is caught by the top-level
Spice environment, which will deal with it in an implementation-specific
fashion.

The {\bf try} expression allows exceptions to be caught and handled.

\begin{tabular}{l}
{\bf def124. TryCatchExpr ::= }\\ 
\hspace*{3mm}{\tt "try" StatementSeq CatchSeq "endtry"} \\ 
\end{tabular}

\begin{tabular}{l}
{\bf def125. CatchSeq ::= }\\ 
\hspace*{3mm}{\tt \begin{tabular}{l}
("catch" Arglist {[}("as"){]} StatementSeq)** \\ 
\end{tabular}
} \\ 
\end{tabular}

The {\bf StatementSeq} of the {\bf try} is evaluated and the {\bf try} expression resturns
its result if no exception is thrown. If an exception {\bf E} is thrown, then
the {\bf catch} body who's {\bf Arglist} best matches the thrown value is executed
(just as though the {\bf catch} blocks were alternative definitions for an
overloaded function) and the reult of the {\bf try} is the result of that
{\bf catch} block.

If there is no {\bf catch (x: Any)} catch clause, it is as though the clause
{\bf catch (x:Any) with throw x} had been supplied.

\begin{quote}This is not true; the description needs revising.\end{quote}\begin{quote}Note that this is {\em not} sequential testing. Sequential testing means that you
have to write the general case last, not first, and makes things more
order-dependant than they need be. Making the {\bf catch} clauses share the
semantics of procedure call makes the language more coherent and offers
alternative implementation tactics.\end{quote}\subsection{if}


An {\bf if} expression expresses a choice. Both verbose and compact if-endif forms
are supported; in the short form, {\bf if} - {\bf endif} is replaced by {\bf (} - {\bf )},
"$\mid$" is used in place of {\bf then} and {\bf else}, and "$\mid$:" in place of {\bf elseif}.

An {\bf if} must have a matching {\bf endif}, and an {\bf unless} must have a matching
{\bf endunless}. {\bf then} and {\bf do} are equivalent in {\bf if} and {\bf unless} expressions.
The {\bf unless} form inverts the initial test.

\begin{tabular}{l}
{\bf def126. IfExpr ::= }\\ 
\hspace*{3mm}{\tt LongIfExpr} \\ 
\hspace*{3mm}{\tt  $\mid$ ShortIfExpr} \\ 
\end{tabular}

\begin{tabular}{l}
{\bf def127. ShortIfExpr ::= }\\ 
\hspace*{3mm}{\tt \begin{tabular}{l}
"(" \\ 
 Expr \\ 
 "$\mid$" \\ 
 CommaExpr \\ 
 ("$\mid$:" Expr "$\mid$" CommaExpr)* \\ 
 {[}("$\mid$" CommaExpr){]} \\ 
 ")" \\ 
\end{tabular}
} \\ 
\end{tabular}

\begin{tabular}{l}
{\bf def128. LongIfExpr ::= }\\ 
\hspace*{3mm}{\tt \begin{tabular}{l}
("if" $\mid$ "unless") \\ 
 Expr \\ 
 Then \\ 
 StatementSeq \\ 
 ("elseif" Expr Then StatementSeq)* \\ 
 {[}("else" StatementSeq){]} \\ 
 ("endif" $\mid$ "endunless") \\ 
\end{tabular}
} \\ 
\end{tabular}

\begin{tabular}{l}
{\bf def129. Then ::= }\\ 
\hspace*{3mm}{\tt "then"} \\ 
\hspace*{3mm}{\tt  $\mid$ "do"} \\ 
\end{tabular}

\subsection{repeat}


A {\bf repeat} loop repeats statements while (or until) a condition is true.
On each iteratation, {\bf Seq1} is executed, and then {\bf Expr1} is tested.
If it is satisfied ({\bf true} for {\bf while}, {\bf false}for {\bf until}) then {\bf Seq2} is
executed and the loop repeats. Otherwise, the loop terminates and delivers all
the values computed by the loop body.

\begin{tabular}{l}
{\bf def130. RepeatExpr ::= }\\ 
\hspace*{3mm}{\tt \begin{tabular}{l}
"repeat" \\ 
 StatementSeq1 \\ 
 {[}(RepeatTest {[}("do" StatementSeq2){]}){]} \\ 
 "endrepeat" \\ 
\end{tabular}
} \\ 
\end{tabular}

\begin{tabular}{l}
{\bf def131. RepeatTest ::= }\\ 
\hspace*{3mm}{\tt ("while" $\mid$ "until") Expr1 {[}("then" Expr2){]}} \\ 
\end{tabular}

\subsection{for}


A {\bf for} loop causes a body of code to be repeatedly executed while various
conditions are {\bf true} (or {\bf false}), and variables are stepped "in parallel"
along a sequence of values. When the loop terminates "naturally" (that is,
by exhaustion of one of the sequences) then a termination clause is executed.

\begin{tabular}{l}
{\bf def132. ForExpr ::= }\\ 
\hspace*{3mm}{\tt \begin{tabular}{l}
"for" \\ 
 ForControls \\ 
 {[}("do" StatementSeq){]} \\ 
 {[}("finally" StatementSeq){]} \\ 
 "endfor" \\ 
\end{tabular}
} \\ 
\end{tabular}

The controls are separated by {\bf also}, although that's optional if the next
control is a {\bf while} or an {\bf until}.

\begin{tabular}{l}
{\bf def133. ForControls ::= }\\ 
\hspace*{3mm}{\tt ForControl} \\ 
\hspace*{3mm}{\tt  $\mid$ ForControls "also" ForControl} \\ 
\hspace*{3mm}{\tt  $\mid$ ForControls ForCondition} \\ 
\end{tabular}

A {\bf Control} is a binding of a variable to values, or an early-exit condition.

\begin{tabular}{l}
{\bf def134. ForControl ::= }\\ 
\hspace*{3mm}{\tt ForBinding} \\ 
\hspace*{3mm}{\tt  $\mid$ ForCondition} \\ 
\end{tabular}

If the early-exit condition is satisfied ({\bf while false} or {\bf until true}),
the loop terminates. If the {\bf then} (for {\bf until}) or {\bf else} (for {\bf while})
is present, its code is executed before the loop exits; otherwise the loop's
{\bf finally} code is executed.

\begin{tabular}{l}
{\bf def135. ForCondition ::= }\\ 
\hspace*{3mm}{\tt \begin{tabular}{l}
("while" $\mid$ "until") \\ 
 Expr \\ 
 {[}(("then" $\mid$ "else") CommaExpr){]} \\ 
\end{tabular}
} \\ 
\end{tabular}

A {\bf ForBinding} binds names to sequences of values.

\begin{tabular}{l}
{\bf def136. ForBinding ::= }\\ 
\hspace*{3mm}{\tt ForName "in" Expr} \\ 
\hspace*{3mm}{\tt  $\mid$ \begin{tabular}{l}
ForName \\ 
 "from" \\ 
 Expr \\ 
 {[}(("to" $\mid$ "downto") Expr){]} \\ 
\end{tabular}
} \\ 
\hspace*{3mm}{\tt  $\mid$ ForName "to" Expr} \\ 
\end{tabular}

\begin{tabular}{l}
{\bf def137. ForName ::= }\\ 
\hspace*{3mm}{\tt Name {[}(TypeWord Type){]}} \\ 
\end{tabular}

For-loop identifiers are declared there, are immutable, and are local
to the loop. What is more, a {\em new} identifier is bound each time round
the loop (this is noticable only if the loop body forms closures using
lambda-expressions or holes which involve the identifiers).

The meaning of a {\bf ForBindings} is that each {\bf Name} introduced is given
successive values from its {\bf Expr}, in parallel (ie on the {\bf N}th iteration each
{\bf Name} has its {\bf N}th value), until at least one of the {\bf Expr}s is exhausted.
The notion of "successive values" is defined by the run-time type of the
expression's value. It is not defined what happens if that value is modified
during the execution of the loop body.

The form {\bf from E$_ 1$ to E$_ 2$} is equivalent to {\bf in {[}E$_ 1$..E$_ 2${]}}. If {\bf to E} is
omitted, then the loop is unbounded (it will terminate only via {\bf break} or if
a parallel iteration terminates). If {\bf from E} is omitted, the loop starts at
{\bf 1}. If {\bf downto} is used instead of {\bf to}, the loop counts down rather than up.\subsection{switch}


\begin{tabular}{l}
{\bf def138. SwitchExpr ::= }\\ 
\hspace*{3mm}{\tt "switch" Expr Case+ "endswitch"} \\ 
\end{tabular}

\begin{tabular}{l}
{\bf def139. Case ::= }\\ 
\hspace*{3mm}{\tt ("case" Expr)+ "then" StatementSeq} \\ 
\hspace*{3mm}{\tt  $\mid$ "else" StatementSeq} \\ 
\end{tabular}

Each {\bf case} clause starts with some number of {\bf case} expressions (or
{\bf else}) and is followed by the expressions to evaluate when the switch
expression takes the value of one of those labels.

A Spice switch may switch on {\em any} values; it is not restricted to integer
constants (although it may be much more efficient on them). In particular, a
Spice switch may switch on strings and symbols.

A {\bf switch} can have at most one {\bf else} clause.

The result from a {\bf switch} is the result from the selected {\bf case} statement or
statement sequence. Thus a {\bf switch} may deliver multiple values.\subsection{var and procedure expressions}


A {\bf VarDef} and a {\bf ProcedureDef} counts as expressions, but they deliver
no values.

Note that this means that procedure definitions may be nested. A
procedure may access and update the local variables of the procedure that it
appears in; Spice has full lexical scope.\section{type expressions}


A {\bf type expression} is an expression. Its semantics is the same as that of
ordinary expressions, but they must deliver values of type {\bf Type}.

\begin{tabular}{l}
{\bf def140. Type ::= }\\ 
\hspace*{3mm}{\tt Expr} \\ 
\end{tabular}

The language of type expressions is intended to allow the programmer
to give the compiler useful information about the program.

The basic type expression is the {\em name}. A name used as a type
expression should be the name of one of the built-in types or a
class. {\bf Any} is the name of the universal type, {\bf Object} is the
name of all object types.

\begin{quote}Only names are legal type-expressions in this edition of the Spice
compiler.\end{quote}If {\bf T} is a type-expression, then {\bf T??} represents the {\em optional type}
of {\bf T}; it is the type of values which may be {\bf T} or may be the value
{\bf absent}. {\bf absent} is also spelt {\bf null}.

If {\bf T} and {\bf U} are type-expressions, so is {\bf T,U}, which represents the
type of multiple values with first component(s) or type {\bf T} and second
component(s) of type {\bf U}.

If {\bf T} is a type expression then so is {\bf T**}, the type of lots of
multiple values all of type {\bf T}.

If {\bf T} is a type-expression, {\bf \{T\}} ("array of T", "row of T") is the
type of arrays who's elements are required to be of type {\bf T}.\chapter{scope rules}


The scope rules for Spice define where identifiers are visible. They are
intended to be natural and in most cases are straightforward.

A sequence of forms (eg, a package body, or a non-package)
consists of sequences of consecutive declarations, {\em bunches}, interspersed
with non-definition expressions. All the {\bf Definition}s in a bunch are mutually
in scope. In consequence, there is rarely any need for "forward" declarations.
Non-definitions terminate this mutual scope to allow the rules for interactive
execution to be the same as those for batch compilation.

{\bf class}es do not introduce new scopes; the declarations within a class put
names into the top-level namespace. The same name cannot legally declared more
than once. (A {\bf method} definition does not re-declare a name; it attaches
additional definitions to an existing generic procedure.) Class slots must be
uniquely named in a scope.

A procedure definition introduces a new scope; the arguments and the body
share that scope. (Thus is it illegal to have two arguments with the same
name, or an argument to have the same name as a top-level local of the
procedure.) The body of the procedure is bunched in the same way as a package.

A {\bf LambdaExpr} introduces a new scope. All the names from the surrounding
scope are visible inside the lambda, with the same rights; in particular,
it is permitted to assign to locals of the surrounding context, and for this
to "work": Spice has full lexical scoping.

A {\bf try} expression introduces a new scope which extends to the first {\bf catch}
clause (or to the {\bf endtry} if there are no explicit {\bf catch} clauses). A
{\bf catch} clause introduces a new scope which ends at the next {[}non-nested!{]}
{\bf catch} or {\bf endtry}.

Each of the arms of an {\bf if} or {\bf unless} expression have their own scopes. The
body of a {\bf repeat} loop is a scope. (What's more, variables declared
in the body of a loop are re-declared {\em each time round the loop}.)

Each {\bf SpiceCase} introduces a new scope which ends at the end of its
{\bf StatementSeq}.

A {\bf for-do} statement introduces a new scope. The identifiers declared in
a {\bf ForBinding} are in scope in any succeeding {\bf while} or {\bf until} clause, and
in the loop body, but {\em not} in the {\bf Expr} parts of other {\bf ForBinding}s of this
loop. As for {\bf while} and {\bf until} loops, these identifiers are re-declared
each time round the loop.

The right operand of an {\bf and} or {\bf or} operator has its own scope. (This odd
rule is because {\bf x and ((var y = 42), z)} is a legal expression, but then
questions would arise about {\bf y}'s existence or value if {\bf x} were {\bf false}.)\chapter{the standard library}


Spice has a large library of standard values and procedures. Many of these
have not been implemented yet.

\section{generic procedures}


\begin{itemize}\item {\bf x.typeof}: the type of {\bf x}.

\item {\bf x.printOn( s )}:
    send a human-readable representation of x down the stream s.
    Characters are sent as-is, integers are printed as their signed
    decimal representation {[}note: there ought to be a way of defining
    a negative numeric literal{]}, strings are printed as their
    sequence of characters, arrays are printed as {[}elem1 elem2 ...{]},
    and objects are printed as defined by their overloading of
    printOn (and Class.printOn( s ) does something sensible).

\item {\bf x.print}
    shorthand for x.printOn( standardOutput )

\item {\bf x.report}:
    shorthand for x.printOn( standardReport )

\item {\bf println( x... )}: {\bf print} all the arguments in turn, prefixing all
but the first with a space, and printing a newline after printing
all the arguments.

\item {\bf x.copy}:
    make a copy of x. If x is atomic (Small, Symbol, Char, Procedure)
    just delivers x. If x is composite object, defaults to making
    a shallow copy of x. Overloadable for new types.

\item {\bf x.toString}:
    convert {\bf x} to a string in the same way that {\bf print} does.

\item {\bf equals( x, y )}
    the procedure implementing =. Overloadable on new types, but
    the programmer has to ensure that the new definition respects
    the usual rules for equality (reflexive, transitive, symmetric).

\item If x and y are both Small, Symbol, Char, or Procedure, equals
    delivers x == y (ie identity). If x == y, equals delivers
    true.

\item If x and y are both objects, by default equals does a slot-by-
    slot = test on all the init-slots that x and y have in common.
    If any of these fail, equals delivers false; otherwise it
    delivers true.

\item If x and y are both String, equals does the obvious string
    equality test. If they are both arrays, equals does the obvious
    sequence-equality test.

\item {\bf x.hashCode}:
    a type-specific hashcode for the value {\bf x}.

\item {\bf apply( f, x... )}:
    called when evaluating f(X) and f is not a function. Overloadable.
    On arrays and strings is equivalent to indexing. On tables is
    table lookup. Has the obvious updater. Allows objects to represent
    functions compactly. Is not implemented.

\item {\bf x.length}:
    overloadable. Should deliver the "length" of the object. On
    strings and arrays has the obvious meaning. On objects delivers
    the number of "obvious" slots (see explode). On atomic objects
    delivers 0.

\item {\bf x @asType t}:
    {\bf t} must be a type. The value {\bf x} is converted to that type in a
    type-specific way ({\bf asType} is overloadable), or an exception is thrown.
    {\bf x @asType String} is equivalent to {\bf toString}.

\end{itemize}\section{absent}


The special value {\bf absent} corresponds roughly to the value {\bf null} in
other programming languages: it denotes a "missing" value.

There are two special operators for use with {\bf absent}, corresponding to the
boolean operators {\bf and} and {\bf or}.

The expression {\bf X $\mid$$\mid$ Y} is the value of {\bf X}, unless it is {\bf absent}, in which
case it is the value of {\bf Y}. Only if {\bf X} is absent is {\bf Y} evaluated. {\bf X} and
{\bf Y} should both be expressions of arity 1. {\bf $\mid$$\mid$} can usefully be thought of
as "if absent".

The expression {\bf X \&\& Y} is {\bf absent} if {\bf X} is {\bf absent}; otherwise it is the
value of {\bf Y}. {\bf Y} is not evaluated if {\bf X} is {\bf absent}. {\bf \&\&} can usefully be
thought of as "if present".\section{numbers}


Spice has several kinds of number: small integers ({\bf Small}), big integers
({\bf BigInt}), short and long floating point values ({\bf Float}, {\bf Double}),
ratios (ie fractions) ({\bf Ratio}), and complex numbers ({\bf Complex}).

\begin{quote}Only {\bf Small} and {\bf Double} are implemented.\end{quote}The operators {\bf +}, {\bf -}, {\bf *}, {\bf /}, {\bf div}, {\bf rem}, {\bf \%} work more-or-less as
you'd expect on plain numbers (they work on some other values, too, for which
see the section on {\bf units}). {\bf div} is integer division; the operands must be
integers and the result is truncated toward {\bf 0}. {\bf rem} is integer remainder;
the operands must be integers and the result is the appropriate remainder. {\bf /}
will produce a floating-point value if the result is not an exact integer.

The operator {\bf /:} is {\em exact division}; its result uses ratios to express
non-floating non-integer components. This is the primary way (the only way, at
root) that ratio values are generated.

The operator {\bf ..} ("upto") delivers multiple results: all the integers
from the left operand (inclusive) to the right operand (also
inclusive).

There are no bitwise operators; instead Spice provides them as predefined
functions (to allow the operator symbols to be used for other purposes).
The functions provided are {\bf bitAnd}, {\bf bitXor}, {\bf bitOr}, {\bf bitNot}, {\bf bitBic},
{\bf bitShl}, {\bf bitShr}, and {\bf bitAsr}, operating on integer arguments only.

The relational operators {\bf $<$=}, {\bf $<$}, {\bf $>$=}, {\bf $>$}, {\bf ==}, {\bf /==} work as you'd
expect. Note that it's unwise to use {\bf ==} and {\bf /==} on floating-point values.

There are also a number of standard functions.

\begin{itemize}\item {\bf x.abs} and its synonym {\bf x.magnitude} is the absolute value of {\bf x}; if {\bf x} is
not complex, its sign is made positive; if {\bf x} is complex, it is its
magnitude.

\item {\bf x.neg} and its equivalent {\bf x.negative} invert the sign of the value
of {\bf x}. {\bf x.positive} returns {\bf x} if it is a (non-complex) number.

\item {\bf x @min y}, {\bf x @max y} deliver the minimum (maximum) of their argument
values {\bf x} and {\bf y}.

\item {\bf x @logToBase y} delivers the logarithm of {\bf x} to the (positive
integer) base {\bf y}.

\item {\bf x @toPower y} delivers {\bf x} raised to the power {\bf y}.

\item {\bf x.sqrt} is the square root of {\bf x}. An exception is thrown if the result
must be {\bf Complex} but this implemention does not support it.

\item {\bf x.round}, {\bf x.floor}, {\bf x.ceiling} convert the value {\bf x} to an integer
(details to be added).

\item {\bf x.cos}, {\bf x.sin}, {\bf x.tan} are the usual trigonometric functions. The
argument is in radians.

\item {\bf x.inRadians} takes an angle in degrees {\bf x} and converts it to
radians. {\bf x.inDegrees} takes an angle in radians and converts it to
degrees.

\item {\bf x.acos}, {\bf x.asin}, {\bf x.atan} are the usual inverse trigonometric
functions. Note that {\bf atan} is only suitable for simple programs.

\end{itemize}{[}various complex functions etc to be done{]}\section{enumerations}


Enumeration values can be queried for their numeric value and name, and can be
created from numeric values or names. Note that the number of enumeration
values is fixed by the declaration; the constructor delivers existing values,
not fresh ones.

\begin{itemize}\item {\bf e.index}: if {\bf e} is an enumeration value, its index is its position
in the list in which it was declared.

\item {\bf new E(n)}: if {\bf E} is an enumeration type, the {\bf n}th enumeration value that it
was declared with, or an error is thrown if {\bf n} is out of range.

\item {\bf e.enumName}: if {\bf e} is an enumeration value, the symbol which names it.

\item {\bf new E(s)}: if {\bf E} is an enumeration type, and {\bf s} is a string or symbol,
the enumeration value of {\bf E} named by that string or symbol, or an error
is thrown.

\end{itemize}\section{values with units}


Spice has values with {\bf units}. Values with units represent lengths, or
times, or other inter-related quantities. All the unit procedures are
imported qualified from {\bf Spice.language.units}.

A {\bf unit} expresses an amount along a given dimension. A {\bf basic dimension} is
represented by a symbol naming that dimension. A {\bf compound dimension} is the
result of a product or quotient of dimensions.

\begin{itemize}\item {\bf Unit}: the type of units.

\item {\bf Unit( scale, unitName, dimension )}: make a new unit object. {\bf dimension} must
be a symbol or string; it specifies the dimension of the unit. (Typical values
are {\bf length}, {\bf time}, {\bf colour}.) {\bf unitName} must be a symbol or string; it
names the unit itself. {[}It is an error if a unit with that name already
exists.{]} The {\bf scale} expresses the scale of this unit in terms of the
"canonical unit" for this kind; a scale of {\bf 1} defines the canonical unit.

\item {\bf U @unitProduct V}: {\bf U} and {\bf V} must be unit objects. The result is the
product of those units.

\item {\bf U @unitQuotient V}: {\bf U} and {\bf V} must be unit objects. The result is the
quotient of those units.

\item {\bf U.scale}: {\bf U} must be a {\bf Unit} object; the result is its scale.

\item {\bf U.name}: {\bf U} must be a {\bf Unit}object; the result it its unit name.

\item {\bf U.dimension}: {\bf U} must be a {\bf Unit} object; the result is its dimension.

\item {\bf x @inUnits y}: {\bf x} must be a numeric value, and {\bf y} must be a unit name
(symbol or string) or a {\bf Unit} object (eg from {\bf unitNamed}). The result is a
unit value with {\bf x} as its number and the unit (named by) {\bf y} as its unit.

\item {\bf n.unitNamed}: the {\bf Unit} value named by {\bf n}, which must be a string or a
symbol.

\item {\bf u.number}: the number of the value {\bf u}, which must be a value with
units.

\item {\bf u.unit}: the unit of the value {\bf u}, which must be a value with units.

\item {\bf u.explode}: {\bf u} must be a value with units; exploding it gets the number
and unit, in that order.

\end{itemize}The arithmetic operators also work on values with units. There is a minor
complication in that different units of the same kind may be inter-converted.
Multiplication and division also work on {\bf Unit} objects themselves to produce
new {\bf Unit} objects; this is intended for use in {\bf define unit} definitions.

\begin{itemize}\item {\bf u + v, u - v}: {\bf u} and {\bf v} must be values with the same dimension. The number
with the larger unit is scaled to the size of the smaller, and the result is
the sum (difference) of the numbers, with the smaller unit as its unit. {[}{\em ie},
adding inches to miles gets a result in inches.{]}

\item {\bf u * v}: if {\bf u} or {\bf v} has units, the result has number {\bf u.number *
v.number} and units {\bf u.unit @unitProduct v.unit}.

\item {\bf U * V}: if {\bf U} and {\bf V} are units or numbers, the result is a new unit which
has dimensions the product of the dimensions of {\bf U} and {\bf V}, and number the
product of the numbers of {\bf U} and {\bf V}.

\item {\bf u / v}: if {\bf u} or {\bf v} has units, the result has number {\bf u.number / v.number}
and units {\bf u.unit @unitQuotient v.unit}.

\item {\bf U / V}: if {\bf U} and {\bf V} are units or numbers, the result is a new unit with
dimensions the quotient of the dimensions of {\bf U} and {\bf V}, and number the
quotient of the numbers of {\bf U} and {\bf V}.

\item {\bf u $<$ v}: equivalent to {\bf number(v - u) $<$ 0}, and similarly for the other
relational operators.

\end{itemize}\section{strings}


Strings are sequences of Unicode characters in canonical form. Strings created
as literal tokens are immutable. Strings, like arrays, can be indexed using
the {\bf s{[}i{]}} notation. The infix operator {\bf ++} concatenates strings (and
arrays).

There are many built-in functions on strings.

\begin{itemize}\item {\bf s.length} is the length in characters of {\bf s}.

\item {\bf s @indexOf x} delivers the index in {\bf s} of the first occurance of
{\bf x}, which can be a character or a string. If {\bf s} does not contain any
occurence of {\bf x}, {\bf indexOf} delivers {\bf absent}. The result has the same
base as the string does.

\item {\bf s @lastIndexOf x} is the same as {\bf indexOf}, except that it searches
for the last occurence of {\bf x}, not the first.

\item {\bf s @chopAt x}, where {\bf x} is a string or character, delivers {\em two}
results {\bf (before, after)} such that {\bf s == (before $<$$>$ x $<$$>$ after)} and
{\bf x} does not occur earlier in {\bf s}.

\item {\bf s @chopAt x}, where {\bf x} is an integer, delivers {\bf (after, before)}
such that {\bf s = before $<$$>$ after} and {\bf after.length == x} -
s.stringBase.

\item {\bf s @split x} delivers an array of strings obtained by repeatedly
chopping {\bf s} with {\bf x}.

\item {\bf s.toLower} is a copy of {\bf s} with all upper-case characters replaced
by their lower-case counterparts; {\bf s.toUpper} is a copy of {\bf s} with
all lower-case characters replaced by their upper-case counterparts.

\item {\bf x @catStringly y}: equivalent to {\bf x $<$$>$ y.toString}.

\end{itemize}\section{symbols}


\begin{quote}Symbols are not implemented.\end{quote}The {\em symbol literal} {\bf `foo`} represents a {\em symbol} (a value
of type {\bf Symbol}) whose {\em spelling} is the string {\bf foo}. The important
difference between symbols and strings is that there is only one
symbol with a given spelling. (You can tell the difference using the
Spice {\em identity operator} {\bf ==}.)

You can index symbols using {\bf s{[}i{]}}, but if you do so, you are probably
using symbols for something other than their intended use; supplying
the programmer with an infinite set of mnenomic values.

\begin{itemize}\item {\bf s.intern}, where {\bf s} is a string, delivers the symbol whose spelling
is {\bf s}, making it if necessary.

\item {\bf s.spelling}, where {\bf s} is a symbol, delivers an immutable string which
is the spelling of {\bf s}.

\item {\bf s.explode}, where {\bf s} is a symbol, delivers all the characters of {\bf s},
rightmost last.

\end{itemize}\section{booleans}


The built-in operators {\bf not}, {\bf and}, {\bf or} on booleans operate as you
might expect; {\bf not} negates its operand, {\bf and} is boolean {\bf and} and
{\bf or} is boolean {\bf or}, neither of which evaluate their second operand
unless it's necessary.\section{arrays}


Array elements can be accessed and updated using the {\bf a{[}i{]}} notation.

\begin{itemize}\item {\bf a.length}: delivers the length of the array {\bf a}.

\item {\bf a @reduceBy (x, p)}: {\bf a} must be an array, {\bf x} some value, and {\bf p} a dyadic
procedure. The result is obtained by starting with {\bf x} and repeatedly
replacing it by the result of {\bf p(result,ai)} for {\bf ai} being bound to all the
elements of {\bf a} in turn.

\item {\bf a @join s, a.join}: equivalent to {\bf a @reduceBy (s, catStringly)}; with one
argument, {\bf s} is taken to be the null string {\bf ""}.

\item {\bf a @sortInPlaceWith f}: sorts the array {\bf a} according to the comparison
function {\bf f}. {\bf f} should accept two arguments {\bf (x, y)} and deliver
{\bf true} if {\bf x} should go before {\bf y} in the ordering and {\bf false}
otherwise. {\bf sortInPlaceWith} is guaranteed to be stable if {\bf f} is
sane.

\item {\bf a @sortWith f}: makes a copy of {\bf a} and sorts that in place with {\bf f},
so far as anyone can tell. (The implementor is at liberty to make a
more efficient version.)

\item {\bf a.reverse}: a copy of {\bf a} with the elements reversed.

\item {\bf a.reverseInPlace}: reverses the order of the elements of {\bf a}.

\end{itemize}\section{bits}


Spice does not have standard operators for bit operations on integer values;
instead, it has standard functions which you call in infix form. This reflects
the designer's belief that bit operations are simply not used enough to
warrent using up useful symbols for them.

\begin{itemize}\item {\bf x@bitNot}: the bitwise complement of {\bf x} and {\bf y}.

\item {\bf x @bitAnd y}: the bitwise {\bf and} of {\bf x} and {\bf y}.

\item {\bf x @bitOr y}: the bitwise {\bf or} of {\bf x} and {\bf y}.

\item {\bf x @bitXor y}: the bitwise {\bf exclusive or} of {\bf x} and {\bf y}.

\item {\bf x @bitClear y}: {\bf x @bitAnd y.bitNot}.

\item {\bf x @bitShl y}: {\bf x} shifted left {\bf y} places.

\item {\bf x @bitShr y}: {\bf x} shifted right {\bf y} places. Note: the sign of {\bf x} is {\em always
preserved} for plain ({\bf Small} or {\bf Big}) integer values.

\end{itemize}\section{procedures}


There are several standard functions on procedures.

\begin{itemize}\item {\bf p.updater}: if {\bf p} is a procedure, then delivers its updater (or {\bf absent} if
if hasn't got one).

\item {\bf p @apply }x: {\bf p} must be a procedure, and {\bf x} a collection of values; applies
{\bf p} to all the values in {\bf x}, ie, is {\bf x.explode.p} except that {\bf x} must be a
compound type.

\end{itemize}The different kinds of procedures can be recognised by predicates.

\begin{itemize}\item {\bf x.isProcedure} is {\bf true} when {\bf x} is any kind of procedure, and {\bf false}
otherwise.

\item {\bf x.isMethod} is {\bf true} if {\bf x} is defined by a {\bf method} definition, and {\bf false}
otherwise;

\item {\bf x.isFunction} if {\bf x} is defined by a {\bf function} definition, and {\bf false}
otherwise;

\item {\bf x.isConstructor} is {\bf true} when {\bf x} has been defined with {\bf method new}, and
{\bf false} otherwise;

\item {\bf x.isLambda} is {\bf true} when {\bf x} is the result of a lambda-expression or
hole-expression, and {\bf false} otherwise.

\end{itemize}\section{dictionaries and tables}


The type {\bf Lookup} is the parent type of a variety of mapping data types; the
{\em simple} mapping types map single values to single values, while the {\em compond}
mapping types map tuples of values to tuples of values. There are no direct
instances of {\bf Lookup}. The type {\bf Maplet} is the type of pairs of (simple)
values. The type {\bf Dictionary} is an extension of {\bf Lookup} that maps {\bf Symbol}s
to values.

\begin{itemize}\item {\bf l.length}: the number of key-value associations stored in the lookup {\bf l}.

\item {\bf l.explode}: the key-value pairs stored in the lookup {\bf l}. Each pair is
represented by an instance of the type {\bf Maplet}.

\item {\bf p @mapOver l}: apply the procedure {\bf p} to each of the key-value pairs in {\bf l}
in turn. {\bf p} will be called with two arguments, being the key and the value.
{\em All the results of {\bf p} will be returned.}

\item {\bf l @hasKey k}: if {\bf l} is an instance of an extension of {\bf Lookup}, {\bf hasKey}
returns {\bf true} if {\bf k} is a key for a non-default value in {\bf l} and {\bf false}
otherwise.

\item {\bf l @fetch k}: if {\bf l} is an instance of an extension of {\bf Lookup}, {\bf fetch} gets
the value bound to the key {\bf k}. If no value is bound, some type-specific
computation is performed. {\bf fetch} has an updater which alters the bound value,
or creates a binding if none already exists.

\item {\bf l @apply k}: if {\bf l} is an instance of an extension of {\bf Lookup}, applying it
runs {\bf fetch}.

\item {\bf m.key}: the key part of {\bf m}, ie, the value which would be looked up.

\item {\bf m.value}: the value part of {\bf m}, ie, the value that would be returned when
looking up the {\bf key} part.

\item {\bf Dictionary()}: constructs an empty {\bf Dictionary} object with default value
{\bf absent}. A {\bf Dictionary} is an extension of {\bf Lookup}.

\item {\bf Dictionary(d)}: constructs an empty {\bf Dictionary} with default value {\bf d}.

\item {\bf Dictionary(d, k$_ 1$, v$_ 1$, ..., k$_ n$, v$_ n$)}: constructs a {\bf Dictionary} with
default value {\bf d} and bindings which bind k$_ i$ to v$_ i$. If several k$_ i$ are
equal, the last one wins. All the k$_ i$ must be {\bf Symbol}s.

\item {\bf Table}: an extension of {\bf Lookup} with constructors in the same style as
{\bf Dictionary}. A {\bf Table} can have key arguments of any type; they are compared
using {\bf ==} (and hashed using {\bf hashCode}, qv).

\item {\bf FatTable}: an compound mapping extension of {\bf Lookup}. {\bf FatTable} has similar
constructors to {\bf Dictionary}, except that the keys and values must all be
arrays; all the keys must have the same length, and all the values must have
the same length. A {\bf FatTable} maps tuples of values to tuples of values.

\item {\bf f @fetch (k$_ 1$, ..., k$_ n$)}: if {\bf f} is a {\bf FatTable}, look up the value(s)
specified by the sequence of keys {\bf k$_ i$}, which must be the same length as the
key arrays used in the constructor of {\bf f}. Deliver the associated multiple
values. The updater of {\bf fetch} will store multiple values into {\bf f}.

\end{itemize}\section{io}


Spice includes some simple I/O operations in its core; the type {\bf Pathname},
the type {\bf File}, and the type {\bf Stream}.

\begin{itemize}\item {\bf s.parsePath}: {\bf s} must be a string or symbol. It is parsed into a {\bf Pathname},
or an exception thrown if it is illegal in some way. (See below for the syntax
of pathnames).

\item {\bf p.pathScheme}, {\bf p.pathRoot}, {\bf p.pathDirs}, {\bf p.pathName}, {\bf p.pathSuffix}, and
{\bf p.pathType}: the corresponding components of the pathname. If the component
was omitted in the originating string, the result is {\bf absent}.

\item {\bf p @openIn}: {\bf p} must be a pathname (or string or symbol, which is immediatley
parsed into a pathname). The external entity named by the path is opened for
input. {\bf openIn} delivers a {\bf File} object.

\item {\bf p @openOut}: {\bf p} must be a pathname (or string or symbol, which is immediatley
parsed into a pathname). The external entity named by the path is opened for
output. {\bf openOut} delivers a {\bf File} object.

\item {\bf f.openedOn}: if {\bf f} is a file, then the pathname it was opened on.

\item {\bf f.close}: {\bf f} must be a file; if it is not closed, it is hereby closed,
committing all writes (if open for output).

\item {\bf f @readInto (b, w, l)}: {\bf f} must be a file open for input. {\bf readInto} reads
the next {\bf l} bytes (or fewer) into the buffer {\bf b}, which must be a byte array,
starting at position {\bf w}. An exception is thrown if {\bf w} or {\bf w+l} would be
outside the bounds of {\bf b}. The number of bytes actually read is returned as
the result.

\item {\bf f @writeFrom (b, w, l)}: {\bf f} must be a file open for output. {\bf l} bytes from
the byte array {\bf b} are written, starting at offset {\bf w}. An exception is thrown
if {\bf w} or {\bf w+l} would be outside the bounds of {\bf b}.

\item {\bf f.inFrom}: {\bf f} is a file open for input, or a pathname (which is immediately
{\bf openIn}'ed). The result is an byte {\bf InStream} on the file.

\item {\bf f.outTo}: {\bf f} is a file open for output, or a pathname (which is immediately
{\bf openOut}'ed). The result is an byte {\bf OutStream} on the file.

\item {\bf s.next}: {\bf s} must be an {\bf InStream}; the result is the next object from {\bf s},
or {\bf absent} if the file has been exhausted. {\bf next} has an updater which puts
objects back onto the stream. Arbitrarily many objects can be put back, and
they need not correspond to objects originally present in the stream.

\item {\bf s @out x}: {\bf s} must be an {\bf OutStream} and {\bf x} an object; that object is
appended to the stream.

\item {\bf s.streamFile}: if {\bf s} is a stream based on a file, delivers that file;
otherwise an exception is thrown.

\end{itemize}\subsection{pathname syntax and accessors}


A pathname has several components. The basic elements are {\bf Word}s which,
within this section, are sequences of characters not otherwise reserved
to the pathname syntax.

In this section, the identifier {\bf p} is presumed to contain a {\bf Pathname} value.

\begin{tabular}{l}
{\bf def141. Pathname ::= }\\ 
\hspace*{3mm}{\tt \begin{tabular}{l}
{[}(Scheme){]} \\ 
 {[}(Root){]} \\ 
 {[}(Directories){]} \\ 
 {[}(Name){]} \\ 
 {[}(Suffix){]} \\ 
 {[}(Type){]} \\ 
\end{tabular}
} \\ 
\end{tabular}

\begin{tabular}{l}
{\bf def142. Scheme ::= }\\ 
\hspace*{3mm}{\tt Word ":"} \\ 
\end{tabular}

The {\bf Scheme} of a pathname directs how the components are to be interpreted.
The standard schemes are {\bf file}, {\bf http}, and {\bf socket}.

{\bf p.pathScheme} is a {\bf Symbol}.

\begin{tabular}{l}
{\bf def143. Root ::= }\\ 
\hspace*{3mm}{\tt "//" Dotted} \\ 
\end{tabular}

The {\bf Root} identifies where the pathname is anchored. For an {\bf http} or {\bf net}
pathname, the {\bf Root} is the IP address where the named entity is located. For
a {\bf file} scheme, the {\bf Unit} identifies some root in the filing system in
an implementation-specific way.

{\bf p.pathRoot} is an array of {\bf Symbol} values, one for each word-part of the
{\bf Dotted}.

\begin{tabular}{l}
{\bf def144. Directories ::= }\\ 
\hspace*{3mm}{\tt {[}("/"){]} (Dotted "/")++} \\ 
\end{tabular}

The {\bf Directories} identify some place within the {\bf Root} where an entity is
found, by giving a sequence of {\bf Dotted}s. For a {\bf file} scheme, thet are the
names of filing-system directories. For an {\bf http} scheme, they are successive
components of the directory part of the URL. For a {\bf socket} scheme, they
identify the port number to be used for the socket; typically there is but one
{\bf Dotted} and it names an IP service.

{\bf p.pathDirectories} is an array of {\bf Symbol}s, each symbol being the spelling
of a {\bf Dotted}.

\begin{tabular}{l}
{\bf def145. Name ::= }\\ 
\hspace*{3mm}{\tt Word} \\ 
\end{tabular}

The {\bf Name} is the leafname of the entity being described.

{\bf p.pathName} is a Symbol.

\begin{tabular}{l}
{\bf def146. Suffix ::= }\\ 
\hspace*{3mm}{\tt "." Word} \\ 
\end{tabular}

The {\bf Suffix} is the suffix part of the entity name.

{\bf p.pathSuffix} is a {\bf Symbol}.

\begin{tabular}{l}
{\bf def147. Type ::= }\\ 
\hspace*{3mm}{\tt ";" Word} \\ 
\end{tabular}

The {\bf Type} is the type of the entity. When a pathname is being used to create
an object, is specifies the type of object to create. When it is being used
to access an existing object, the object should be of a compatible type.

{\bf p.pathType} is a {\bf Symbol}.

\begin{tabular}{l}
{\bf def148. Dotted ::= }\\ 
\hspace*{3mm}{\tt Word++ "."} \\ 
\end{tabular}

\section{other undefined library support}


Spice is expected to support useful library functions for locales, dates, and
regular expressions, but these have not yet been specified.\section{types}


The standard types are

\begin{itemize}\item {\bf Lookup}, the parent type of data dictionaries.

\item {\bf Maplet}, a helper type for data dictionaries.

\item {\bf Table}, the type of tables.

\item {\bf FatTable}, the type of compound tables.

\item {\bf Small}, the type of small integers.

\item {\bf Number}, the type of all numbers -- integers, floats, rationals (when
implemented), complexes (ditto).

\item {\bf Complex}, the type of complex values.

\item {\bf Ratio}, the type of ratios.

\item {\bf BigInt}, the type of big integers.

\item {\bf Float}, the type of all floating-point numbers.

\item {\bf String}, the type of strings.

\item {\bf Procedure}, the type of all procedures.

\item {\bf Bool}, the type of booleans.

\item {\bf Char}, the type of characters.

\item {\bf Any}, the type of anything whatsoever.

\item {\bf Object}, the type of all objects (ie things defined by classes).

\item {\bf Date}, the type of dates.

\item {\bf Unit}, the type of units.

\item {\bf Pathname}, the type of pathname objects used to represent URLs (and local
file names, etc).

\end{itemize}\chapter{spice standard packages}


Spice compes with a collection of standard packages. Most of them are
automaticlly imported by the compiler (except in limited implementations), but
it is permitted to export them explicitly.

\begin{tabular}{|llllllllllllllllllllllllllllllllllllllllllll|}
{\bf package name}          &  &{\bf description} \\ 
Spice.core.complex      &  &handling complex arithmetic \\ 
Spice.core.ratios       &  &handling rational numbers \\ 
Spice.core.bignums      &  &handling big integers \\ 
Spice.core.styles       &  &style operations \\ 
Spice.core.units        &  &{\bf define unit} definitions and operations \\ 
Spice.core.properties   &  &{\bf property} declarations and operations \\ 
Spice.core.multimethods  &  &support for multimethods \\ 
Spice.core.multivalues  &  &support for multiple values \\ 
Spice.core.monomethods  &  &support for monomethods (single-argument
polymorphism) \\ 
Spice.core.multiheritance  &  &support for multiple inheritance \\ 
Spice.core.dates        &  &the {\bf Date} type and its operations \\ 
Spice.core.io           &
 &basic input/output support, including pathnames \\ 
Spice.core.files        &  &file input/output support \\ 
Spice.core.net          &  &network input/output support \\ 
Spice.core.tables       &  &lookup table support \\ 
\end{tabular}
\chapter{glossary}


\begin{itemize}\item {\bf absent}. The "missing" value. Also spelt {\bf null}.

\item {\bf all}. reserved word used following {\bf do} in loops to specify that the value of
the loop includes all the values from iterating through its body.

\item {\bf Any}. The universal type; all values are compatible with {\bf Any}.

\item {\bf class}. A value representing a template for a collection of other values
called its {\bf instances}.

\item {\bf explode}. (a) the name of a procedure which explodes its single argument. (b)
to take a value and deliver all of its consituent values.

\item {\bf enum}. An abbreviated {\bf class} definition which describes a new class and all
of its values.

\item {\bf hole}. The marker {\bf ?} in an expression, representing a parameter position
for an implicit lambda-expression wrapped round the smallest enclosing
application.

\item {\bf initialiser}. A procedure declared with {\bf define init} which initialises
its first argument (usually called {\bf this}) according to the values of its
remaining instances.

\item {\bf lambda expression}. An expression describing a procedure by giving its
arguments and body.

\item {\bf method}. A {\bf procedure} expected to be called using dot-notation and defined
within a {\bf class}. Methods have a usually-implicit first argument usually
called {\bf this}.

\item {\bf measure}. A value consisting of a number and a unit, meaning that many of
that unit.

\item {\bf multiple values}. A Spice expression can evaluate to zero or more values,
and how many values is determined at run-time, not compile-time. Thus a
function call, or a loop, may deliver several results, which may be assigned
or embedded into a larger expression.

\item {\bf new}. Syntactic form for building instances of a class.

\item {\bf null}. Another name for {\bf absent}.

\item {\bf package}. A named collection of definitions and executable expressions; the
unit of encapsulation.

\item {\bf predicate}. A {\bf procedure} of one argument that returns a single {\bf Bool} value.

\item {\bf procedure}. A piece of code invoked from elsewhere to perform some specified
task. A procedure takes some number of arguments, performs some action,
and delivers some number of results.

\item {\bf property}. (a) part of a {\bf style} that gives a value to some aspect of
rendering. (b) the name of such a part. (c) A procedure which maps objects to
values in a way that respects their class inheritance structure. (d) The name
of such a procedure.

\item {\bf slot}. (a) A named location within (instances of) a class. (b) the procedure
which gives access to such a location.

\item {\bf super}. Used in method calls to invoke the "next more general" method.

\item {\bf unit}. The value used in a {\bf measure} to specify the units of that measure,
eg inches, seconds, x's, and so on.

\item {\bf updater}. The part of a {\bf procedure} that implements its behaviour when it is
called as the target of an assignment.

\end{itemize}\chapter{syntactic summary}


This section gathers together all the syntax from the rest of the document.

\begin{tabular}{l}
{\bf def1. Program ::= }\\ 
\hspace*{3mm}{\tt Spice (Package* $\mid$ PackageBody)} \\ 
\end{tabular}

\begin{tabular}{l}
{\bf def2. Spice ::= }\\ 
\hspace*{3mm}{\tt "spice" String ("," (Name ":" Expr))*} \\ 
\end{tabular}

\begin{tabular}{l}
{\bf def3. OpenModifier ::= }\\ 
\hspace*{3mm}{\tt MarkedModifier Modifier*} \\ 
\end{tabular}

\begin{tabular}{l}
{\bf def4. Modifier ::= }\\ 
\hspace*{3mm}{\tt MarkedModifier} \\ 
\hspace*{3mm}{\tt  $\mid$ Name {[}(ModArgument){]}} \\ 
\hspace*{3mm}{\tt  $\mid$ "{[}" Modifier** "," "{]}"} \\ 
\end{tabular}

\begin{tabular}{l}
{\bf def5. MarkedModifier ::= }\\ 
\hspace*{3mm}{\tt "public"} \\ 
\hspace*{3mm}{\tt  $\mid$ "private"} \\ 
\hspace*{3mm}{\tt  $\mid$ "protected"} \\ 
\end{tabular}

\begin{tabular}{l}
{\bf def6. MarkedArgument ::= }\\ 
\hspace*{3mm}{\tt Literal} \\ 
\hspace*{3mm}{\tt  $\mid$ "(" Expression ")"} \\ 
\end{tabular}

\begin{tabular}{l}
{\bf def7. Package ::= }\\ 
\hspace*{3mm}{\tt \begin{tabular}{l}
MarkedModifier \\ 
 "package" \\ 
 PackageName \\ 
 PackageBody \\ 
 {[}("endpackage"){]} \\ 
\end{tabular}
} \\ 
\end{tabular}

\begin{tabular}{l}
{\bf def8. PackageName ::= }\\ 
\hspace*{3mm}{\tt Word++ "."} \\ 
\end{tabular}

\begin{tabular}{l}
{\bf def9. PackageBody ::= }\\ 
\hspace*{3mm}{\tt Import** SEMI Bundle** SEMI} \\ 
\end{tabular}

\begin{tabular}{l}
{\bf def10. Bundle ::= }\\ 
\hspace*{3mm}{\tt Definition+ Expr*} \\ 
\end{tabular}

\begin{tabular}{l}
{\bf def11. Definition ::= }\\ 
\hspace*{3mm}{\tt ProcedureDef} \\ 
\hspace*{3mm}{\tt  $\mid$ VarDef} \\ 
\hspace*{3mm}{\tt  $\mid$ ClassDef} \\ 
\hspace*{3mm}{\tt  $\mid$ UnitDef} \\ 
\hspace*{3mm}{\tt  $\mid$ EnumDef} \\ 
\end{tabular}

\begin{tabular}{l}
{\bf def12. Import ::= }\\ 
\hspace*{3mm}{\tt \begin{tabular}{l}
MarkedModifier \\ 
 "import" \\ 
 OpenModifier \\ 
 {[}(Alias "="){]} \\ 
 PackageName \\ 
 {[}("facet" (Name** ",")){]} \\ 
 {[}("from" Expr){]} \\ 
\end{tabular}
} \\ 
\end{tabular}

\begin{tabular}{l}
{\bf def13. ProcedureDef ::= }\\ 
\hspace*{3mm}{\tt FullProcedureDef} \\ 
\hspace*{3mm}{\tt  $\mid$ FunctionDef} \\ 
\hspace*{3mm}{\tt  $\mid$ CompactDef} \\ 
\end{tabular}

\begin{tabular}{l}
{\bf def14. FullProcedureDef ::= }\\ 
\hspace*{3mm}{\tt \begin{tabular}{l}
"define" \\ 
 Modifier \\ 
 ("method" $\mid$ "function" $\mid$ "generic" $\mid$ "init" $\mid$ "specific") \\ 
 Header \\ 
 {[}("returns" Type){]} \\ 
 {[}("extends" CommaExpr){]} \\ 
 {[}("super" Header){]} \\ 
 ProcedureBody \\ 
 "enddefine" \\ 
\end{tabular}
} \\ 
\end{tabular}

\begin{tabular}{l}
{\bf def15. FunctionDef ::= }\\ 
\hspace*{3mm}{\tt \begin{tabular}{l}
OpenModifier \\ 
 "function" \\ 
 Header \\ 
 {[}("returns" Type){]} \\ 
 {[}("extends" CommaExpr){]} \\ 
 {[}("super" Header){]} \\ 
 ProcedureBody \\ 
 "endfunction" \\ 
\end{tabular}
} \\ 
\end{tabular}

\begin{tabular}{l}
{\bf def16. CompactDef ::= }\\ 
\hspace*{3mm}{\tt "def" Header ("=" $\mid$ "as" $\mid$ "=$>$") Expr} \\ 
\end{tabular}

\begin{tabular}{l}
{\bf def17. Header ::= }\\ 
\hspace*{3mm}{\tt CallShape} \\ 
\hspace*{3mm}{\tt  $\mid$ Name "-$>$" CallShape} \\ 
\hspace*{3mm}{\tt  $\mid$ CallShape ":=" Arg} \\ 
\end{tabular}

\begin{tabular}{l}
{\bf def18. CallShape ::= }\\ 
\hspace*{3mm}{\tt PrefixShape} \\ 
\hspace*{3mm}{\tt  $\mid$ DottedShape} \\ 
\hspace*{3mm}{\tt  $\mid$ InfixShape} \\ 
\end{tabular}

\begin{tabular}{l}
{\bf def19. PrefixShape ::= }\\ 
\hspace*{3mm}{\tt NameA {[}(Arglist){]}} \\ 
\end{tabular}

\begin{tabular}{l}
{\bf def20. DottedShape ::= }\\ 
\hspace*{3mm}{\tt NameB . NameA {[}(Arglist){]}} \\ 
\end{tabular}

\begin{tabular}{l}
{\bf def21. InfixShape ::= }\\ 
\hspace*{3mm}{\tt ArgA @ NameA {[}(ArgB){]}} \\ 
\end{tabular}

\begin{tabular}{l}
{\bf def22. Arglist ::= }\\ 
\hspace*{3mm}{\tt "(" Args ")"} \\ 
\end{tabular}

\begin{tabular}{l}
{\bf def23. Args ::= }\\ 
\hspace*{3mm}{\tt Arg** ","} \\ 
\end{tabular}

\begin{tabular}{l}
{\bf def24. Arg ::= }\\ 
\hspace*{3mm}{\tt Name {[}(":" {[}("!"){]} Type $\mid$ "==" Expr){]}} \\ 
\hspace*{3mm}{\tt  $\mid$ {[}(Name){]} "..."} \\ 
\end{tabular}

\begin{tabular}{l}
{\bf def25. ProcedureBody ::= }\\ 
\hspace*{3mm}{\tt StatementSeq} \\ 
\hspace*{3mm}{\tt  $\mid$ ArgList "=$>$" ProcedureBody} \\ 
\hspace*{3mm}{\tt  $\mid$ Arg "=$>$" ProcedureBody} \\ 
\end{tabular}

\begin{tabular}{l}
{\bf def26. VarDef ::= }\\ 
\hspace*{3mm}{\tt MarkedModifier ("const" $\mid$ "var") NameDecl} \\ 
\end{tabular}

\begin{tabular}{l}
{\bf def27. NameDecl ::= }\\ 
\hspace*{3mm}{\tt \begin{tabular}{l}
(OneDecl $\mid$ "(" OneDecl++ "," ")") \\ 
 {[}(("=" $\mid$ ":=") Expr){]} \\ 
\end{tabular}
} \\ 
\end{tabular}

\begin{tabular}{l}
{\bf def28. OneDecl ::= }\\ 
\hspace*{3mm}{\tt Name {[}(":" Type $\mid$ "..."){]}} \\ 
\end{tabular}

\begin{tabular}{l}
{\bf def29. ClassDef ::= }\\ 
\hspace*{3mm}{\tt \begin{tabular}{l}
"define" \\ 
 Modifier \\ 
 "class" \\ 
 Name \\ 
 {[}("extends" CommaExpr){]} \\ 
 ClassProperty** \\ 
 SEMI \\ 
 "enddefine" \\ 
\end{tabular}
} \\ 
\end{tabular}

\begin{tabular}{l}
{\bf def30. ClassProperty ::= }\\ 
\hspace*{3mm}{\tt Procedure} \\ 
\hspace*{3mm}{\tt  $\mid$ SlotDecl} \\ 
\hspace*{3mm}{\tt  $\mid$ VarDef} \\ 
\end{tabular}

\begin{tabular}{l}
{\bf def31. SlotDecl ::= }\\ 
\hspace*{3mm}{\tt \begin{tabular}{l}
("slot" $\mid$ "shared") \\ 
 Name \\ 
 {[}(":" Type){]} \\ 
 {[}("=" Expr){]} \\ 
 {[}("implements" Name2){]} \\ 
\end{tabular}
} \\ 
\end{tabular}

\begin{tabular}{l}
{\bf def32. UnitDef ::= }\\ 
\hspace*{3mm}{\tt \begin{tabular}{l}
"define" \\ 
 ("unit" $\mid$ "units") \\ 
 UnitDef++ \\ 
 "," \\ 
 "enddefine" \\ 
\end{tabular}
} \\ 
\end{tabular}

\begin{tabular}{l}
{\bf def33. UnitDef ::= }\\ 
\hspace*{3mm}{\tt Word {[}("=" Expr){]}} \\ 
\end{tabular}

\begin{tabular}{l}
{\bf def34. EnumDef ::= }\\ 
\hspace*{3mm}{\tt \begin{tabular}{l}
"define" \\ 
 "enum" \\ 
 {[}(Modifier){]} \\ 
 Name \\ 
 "=" \\ 
 Name++ \\ 
 "," \\ 
 "enddefine" \\ 
\end{tabular}
} \\ 
\end{tabular}

\begin{tabular}{l}
{\bf def35. Statement ::= }\\ 
\hspace*{3mm}{\tt Expr++ "," SEMI} \\ 
\end{tabular}

\begin{tabular}{l}
{\bf def36. StatementSeq ::= }\\ 
\hspace*{3mm}{\tt Statement*} \\ 
\end{tabular}

\begin{tabular}{l}
{\bf def37. Expr ::= }\\ 
\hspace*{3mm}{\tt Name} \\ 
\hspace*{3mm}{\tt  $\mid$ Hole} \\ 
\hspace*{3mm}{\tt  $\mid$ Literal} \\ 
\hspace*{3mm}{\tt  $\mid$ Expr "..."} \\ 
\hspace*{3mm}{\tt  $\mid$ PreOp Expr} \\ 
\hspace*{3mm}{\tt  $\mid$ Expr PostOp} \\ 
\hspace*{3mm}{\tt  $\mid$ Expr InOp Expr} \\ 
\hspace*{3mm}{\tt  $\mid$ Expr AssignOp Expr} \\ 
\hspace*{3mm}{\tt  $\mid$ "(" Expr ")"} \\ 
\hspace*{3mm}{\tt  $\mid$ Expr {[}("." DotExpr){]} {[}(OptExpr){]}} \\ 
\hspace*{3mm}{\tt  $\mid$ Expr {[}("@" DotExpr){]} {[}(OptExpr){]}} \\ 
\hspace*{3mm}{\tt  $\mid$ Expr "{[}" Expr "{]}"} \\ 
\hspace*{3mm}{\tt  $\mid$ "{[}" Expr "{]}"} \\ 
\hspace*{3mm}{\tt  $\mid$ LambdaExpr} \\ 
\hspace*{3mm}{\tt  $\mid$ "new" Name {[}("(" Expr ")"){]}} \\ 
\hspace*{3mm}{\tt  $\mid$ "none" OptExpr} \\ 
\hspace*{3mm}{\tt  $\mid$ "one" Expr} \\ 
\hspace*{3mm}{\tt  $\mid$ "once" Expr} \\ 
\hspace*{3mm}{\tt  $\mid$ "throw" Expr} \\ 
\hspace*{3mm}{\tt  $\mid$ "{[}" CommaExpr "{]}"} \\ 
\hspace*{3mm}{\tt  $\mid$ "\{" CommaExpr "\}"} \\ 
\hspace*{3mm}{\tt  $\mid$ QuoteExpr} \\ 
\hspace*{3mm}{\tt  $\mid$ TryCatchExpr} \\ 
\hspace*{3mm}{\tt  $\mid$ IfExpr} \\ 
\hspace*{3mm}{\tt  $\mid$ RepeatExpr} \\ 
\hspace*{3mm}{\tt  $\mid$ ForExpr} \\ 
\hspace*{3mm}{\tt  $\mid$ SwitchExpr} \\ 
\hspace*{3mm}{\tt  $\mid$ VarDef} \\ 
\hspace*{3mm}{\tt  $\mid$ ProcedureDef} \\ 
\end{tabular}

\begin{tabular}{l}
{\bf def38. OptExpr ::= }\\ 
\hspace*{3mm}{\tt {[}(Expr){]}} \\ 
\end{tabular}

\begin{tabular}{l}
{\bf def39. CommaExpr ::= }\\ 
\hspace*{3mm}{\tt Expr++ ","} \\ 
\end{tabular}

\begin{tabular}{l}
{\bf def40. OptCommaExpr ::= }\\ 
\hspace*{3mm}{\tt {[}(CommaExpr){]}} \\ 
\end{tabular}

\begin{tabular}{l}
{\bf def41. Hole ::= }\\ 
\hspace*{3mm}{\tt "?" {[}(Integer){]}} \\ 
\end{tabular}

\begin{tabular}{l}
{\bf def42. Literal ::= }\\ 
\hspace*{3mm}{\tt StringLiteral} \\ 
\hspace*{3mm}{\tt  $\mid$ NumberLiteral} \\ 
\hspace*{3mm}{\tt  $\mid$ ReservedLiteral} \\ 
\hspace*{3mm}{\tt  $\mid$ CharacterLiteral} \\ 
\end{tabular}

\begin{tabular}{l}
{\bf def43. PostOp ::= }\\ 
\hspace*{3mm}{\tt LexicalPostfixOperator} \\ 
\hspace*{3mm}{\tt  $\mid$ "@" DotExpr} \\ 
\end{tabular}

\begin{tabular}{l}
{\bf def44. InOp ::= }\\ 
\hspace*{3mm}{\tt LexicalInfixOperator} \\ 
\hspace*{3mm}{\tt  $\mid$ "@" DotExpr} \\ 
\end{tabular}

\begin{tabular}{l}
{\bf def45. DotExpr ::= }\\ 
\hspace*{3mm}{\tt Name} \\ 
\hspace*{3mm}{\tt  $\mid$ "(" Expr ")"} \\ 
\hspace*{3mm}{\tt  $\mid$ "return"} \\ 
\hspace*{3mm}{\tt  $\mid$ "throw"} \\ 
\hspace*{3mm}{\tt  $\mid$ "new"} \\ 
\end{tabular}

\begin{tabular}{l}
{\bf def46. LambdaExpr ::= }\\ 
\hspace*{3mm}{\tt "(" Args "=$>$" StatementSeq ")"} \\ 
\hspace*{3mm}{\tt  $\mid$ "fun" Args "=$>$" StatementSeq "endfun"} \\ 
\end{tabular}

\begin{tabular}{l}
{\bf def47. QuoteExpr ::= }\\ 
\hspace*{3mm}{\tt "`" QuotedItems "`"} \\ 
\end{tabular}

\begin{tabular}{l}
{\bf def48. QuotedItems ::= }\\ 
\hspace*{3mm}{\tt QuotedItem*} \\ 
\end{tabular}

\begin{tabular}{l}
{\bf def49. QuotedItem ::= }\\ 
\hspace*{3mm}{\tt Word} \\ 
\hspace*{3mm}{\tt  $\mid$ "\{" QuotedItems "\}"} \\ 
\hspace*{3mm}{\tt  $\mid$ "(" Expr ")"} \\ 
\hspace*{3mm}{\tt  $\mid$ "$^\wedge$" Word} \\ 
\hspace*{3mm}{\tt  $\mid$ "$^\wedge$$^\wedge$" Word} \\ 
\hspace*{3mm}{\tt  $\mid$ Literal} \\ 
\end{tabular}

\begin{tabular}{l}
{\bf def50. TryCatchExpr ::= }\\ 
\hspace*{3mm}{\tt "try" StatementSeq CatchSeq "endtry"} \\ 
\end{tabular}

\begin{tabular}{l}
{\bf def51. CatchSeq ::= }\\ 
\hspace*{3mm}{\tt \begin{tabular}{l}
("catch" Arglist {[}("as"){]} StatementSeq)** \\ 
\end{tabular}
} \\ 
\end{tabular}

\begin{tabular}{l}
{\bf def52. IfExpr ::= }\\ 
\hspace*{3mm}{\tt LongIfExpr} \\ 
\hspace*{3mm}{\tt  $\mid$ ShortIfExpr} \\ 
\end{tabular}

\begin{tabular}{l}
{\bf def53. ShortIfExpr ::= }\\ 
\hspace*{3mm}{\tt \begin{tabular}{l}
"(" \\ 
 Expr \\ 
 "$\mid$" \\ 
 CommaExpr \\ 
 ("$\mid$:" Expr "$\mid$" CommaExpr)* \\ 
 {[}("$\mid$" CommaExpr){]} \\ 
 ")" \\ 
\end{tabular}
} \\ 
\end{tabular}

\begin{tabular}{l}
{\bf def54. LongIfExpr ::= }\\ 
\hspace*{3mm}{\tt \begin{tabular}{l}
("if" $\mid$ "unless") \\ 
 Expr \\ 
 Then \\ 
 StatementSeq \\ 
 ("elseif" Expr Then StatementSeq)* \\ 
 {[}("else" StatementSeq){]} \\ 
 ("endif" $\mid$ "endunless") \\ 
\end{tabular}
} \\ 
\end{tabular}

\begin{tabular}{l}
{\bf def55. Then ::= }\\ 
\hspace*{3mm}{\tt "then"} \\ 
\hspace*{3mm}{\tt  $\mid$ "do"} \\ 
\end{tabular}

\begin{tabular}{l}
{\bf def56. RepeatExpr ::= }\\ 
\hspace*{3mm}{\tt \begin{tabular}{l}
"repeat" \\ 
 StatementSeq1 \\ 
 {[}(RepeatTest {[}("do" StatementSeq2){]}){]} \\ 
 "endrepeat" \\ 
\end{tabular}
} \\ 
\end{tabular}

\begin{tabular}{l}
{\bf def57. RepeatTest ::= }\\ 
\hspace*{3mm}{\tt ("while" $\mid$ "until") Expr1 {[}("then" Expr2){]}} \\ 
\end{tabular}

\begin{tabular}{l}
{\bf def58. ForExpr ::= }\\ 
\hspace*{3mm}{\tt \begin{tabular}{l}
"for" \\ 
 ForControls \\ 
 {[}("do" StatementSeq){]} \\ 
 {[}("finally" StatementSeq){]} \\ 
 "endfor" \\ 
\end{tabular}
} \\ 
\end{tabular}

\begin{tabular}{l}
{\bf def59. ForControls ::= }\\ 
\hspace*{3mm}{\tt ForControl} \\ 
\hspace*{3mm}{\tt  $\mid$ ForControls "also" ForControl} \\ 
\hspace*{3mm}{\tt  $\mid$ ForControls ForCondition} \\ 
\end{tabular}

\begin{tabular}{l}
{\bf def60. ForControl ::= }\\ 
\hspace*{3mm}{\tt ForBinding} \\ 
\hspace*{3mm}{\tt  $\mid$ ForCondition} \\ 
\end{tabular}

\begin{tabular}{l}
{\bf def61. ForCondition ::= }\\ 
\hspace*{3mm}{\tt \begin{tabular}{l}
("while" $\mid$ "until") \\ 
 Expr \\ 
 {[}(("then" $\mid$ "else") CommaExpr){]} \\ 
\end{tabular}
} \\ 
\end{tabular}

\begin{tabular}{l}
{\bf def62. ForBinding ::= }\\ 
\hspace*{3mm}{\tt ForName "in" Expr} \\ 
\hspace*{3mm}{\tt  $\mid$ \begin{tabular}{l}
ForName \\ 
 "from" \\ 
 Expr \\ 
 {[}(("to" $\mid$ "downto") Expr){]} \\ 
\end{tabular}
} \\ 
\hspace*{3mm}{\tt  $\mid$ ForName "to" Expr} \\ 
\end{tabular}

\begin{tabular}{l}
{\bf def63. ForName ::= }\\ 
\hspace*{3mm}{\tt Name {[}(TypeWord Type){]}} \\ 
\end{tabular}

\begin{tabular}{l}
{\bf def64. SwitchExpr ::= }\\ 
\hspace*{3mm}{\tt "switch" Expr Case+ "endswitch"} \\ 
\end{tabular}

\begin{tabular}{l}
{\bf def65. Case ::= }\\ 
\hspace*{3mm}{\tt ("case" Expr)+ "then" StatementSeq} \\ 
\hspace*{3mm}{\tt  $\mid$ "else" StatementSeq} \\ 
\end{tabular}

\begin{tabular}{l}
{\bf def66. Type ::= }\\ 
\hspace*{3mm}{\tt Expr} \\ 
\end{tabular}

\begin{tabular}{l}
{\bf def67. Pathname ::= }\\ 
\hspace*{3mm}{\tt \begin{tabular}{l}
{[}(Scheme){]} \\ 
 {[}(Root){]} \\ 
 {[}(Directories){]} \\ 
 {[}(Name){]} \\ 
 {[}(Suffix){]} \\ 
 {[}(Type){]} \\ 
\end{tabular}
} \\ 
\end{tabular}

\begin{tabular}{l}
{\bf def68. Scheme ::= }\\ 
\hspace*{3mm}{\tt Word ":"} \\ 
\end{tabular}

\begin{tabular}{l}
{\bf def69. Root ::= }\\ 
\hspace*{3mm}{\tt "//" Dotted} \\ 
\end{tabular}

\begin{tabular}{l}
{\bf def70. Directories ::= }\\ 
\hspace*{3mm}{\tt {[}("/"){]} (Dotted "/")++} \\ 
\end{tabular}

\begin{tabular}{l}
{\bf def71. Name ::= }\\ 
\hspace*{3mm}{\tt Word} \\ 
\end{tabular}

\begin{tabular}{l}
{\bf def72. Suffix ::= }\\ 
\hspace*{3mm}{\tt "." Word} \\ 
\end{tabular}

\begin{tabular}{l}
{\bf def73. Type ::= }\\ 
\hspace*{3mm}{\tt ";" Word} \\ 
\end{tabular}

\begin{tabular}{l}
{\bf def74. Dotted ::= }\\ 
\hspace*{3mm}{\tt Word++ "."} \\ 
\end{tabular}

\begin{tabular}{l}
{\bf def75. Program ::= }\\ 
\hspace*{3mm}{\tt Spice (Package* $\mid$ PackageBody)} \\ 
\end{tabular}

\begin{tabular}{l}
{\bf def76. Spice ::= }\\ 
\hspace*{3mm}{\tt "spice" String ("," (Name ":" Expr))*} \\ 
\end{tabular}

\begin{tabular}{l}
{\bf def77. OpenModifier ::= }\\ 
\hspace*{3mm}{\tt MarkedModifier Modifier*} \\ 
\end{tabular}

\begin{tabular}{l}
{\bf def78. Modifier ::= }\\ 
\hspace*{3mm}{\tt MarkedModifier} \\ 
\hspace*{3mm}{\tt  $\mid$ Name {[}(ModArgument){]}} \\ 
\hspace*{3mm}{\tt  $\mid$ "{[}" Modifier** "," "{]}"} \\ 
\end{tabular}

\begin{tabular}{l}
{\bf def79. MarkedModifier ::= }\\ 
\hspace*{3mm}{\tt "public"} \\ 
\hspace*{3mm}{\tt  $\mid$ "private"} \\ 
\hspace*{3mm}{\tt  $\mid$ "protected"} \\ 
\end{tabular}

\begin{tabular}{l}
{\bf def80. MarkedArgument ::= }\\ 
\hspace*{3mm}{\tt Literal} \\ 
\hspace*{3mm}{\tt  $\mid$ "(" Expression ")"} \\ 
\end{tabular}

\begin{tabular}{l}
{\bf def81. Package ::= }\\ 
\hspace*{3mm}{\tt \begin{tabular}{l}
MarkedModifier \\ 
 "package" \\ 
 PackageName \\ 
 PackageBody \\ 
 {[}("endpackage"){]} \\ 
\end{tabular}
} \\ 
\end{tabular}

\begin{tabular}{l}
{\bf def82. PackageName ::= }\\ 
\hspace*{3mm}{\tt Word++ "."} \\ 
\end{tabular}

\begin{tabular}{l}
{\bf def83. PackageBody ::= }\\ 
\hspace*{3mm}{\tt Import** SEMI Bundle** SEMI} \\ 
\end{tabular}

\begin{tabular}{l}
{\bf def84. Bundle ::= }\\ 
\hspace*{3mm}{\tt Definition+ Expr*} \\ 
\end{tabular}

\begin{tabular}{l}
{\bf def85. Definition ::= }\\ 
\hspace*{3mm}{\tt ProcedureDef} \\ 
\hspace*{3mm}{\tt  $\mid$ VarDef} \\ 
\hspace*{3mm}{\tt  $\mid$ ClassDef} \\ 
\hspace*{3mm}{\tt  $\mid$ UnitDef} \\ 
\hspace*{3mm}{\tt  $\mid$ EnumDef} \\ 
\end{tabular}

\begin{tabular}{l}
{\bf def86. Import ::= }\\ 
\hspace*{3mm}{\tt \begin{tabular}{l}
MarkedModifier \\ 
 "import" \\ 
 OpenModifier \\ 
 {[}(Alias "="){]} \\ 
 PackageName \\ 
 {[}("facet" (Name** ",")){]} \\ 
 {[}("from" Expr){]} \\ 
\end{tabular}
} \\ 
\end{tabular}

\begin{tabular}{l}
{\bf def87. ProcedureDef ::= }\\ 
\hspace*{3mm}{\tt FullProcedureDef} \\ 
\hspace*{3mm}{\tt  $\mid$ FunctionDef} \\ 
\hspace*{3mm}{\tt  $\mid$ CompactDef} \\ 
\end{tabular}

\begin{tabular}{l}
{\bf def88. FullProcedureDef ::= }\\ 
\hspace*{3mm}{\tt \begin{tabular}{l}
"define" \\ 
 Modifier \\ 
 ("method" $\mid$ "function" $\mid$ "generic" $\mid$ "init" $\mid$ "specific") \\ 
 Header \\ 
 {[}("returns" Type){]} \\ 
 {[}("extends" CommaExpr){]} \\ 
 {[}("super" Header){]} \\ 
 ProcedureBody \\ 
 "enddefine" \\ 
\end{tabular}
} \\ 
\end{tabular}

\begin{tabular}{l}
{\bf def89. FunctionDef ::= }\\ 
\hspace*{3mm}{\tt \begin{tabular}{l}
OpenModifier \\ 
 "function" \\ 
 Header \\ 
 {[}("returns" Type){]} \\ 
 {[}("extends" CommaExpr){]} \\ 
 {[}("super" Header){]} \\ 
 ProcedureBody \\ 
 "endfunction" \\ 
\end{tabular}
} \\ 
\end{tabular}

\begin{tabular}{l}
{\bf def90. CompactDef ::= }\\ 
\hspace*{3mm}{\tt "def" Header ("=" $\mid$ "as" $\mid$ "=$>$") Expr} \\ 
\end{tabular}

\begin{tabular}{l}
{\bf def91. Header ::= }\\ 
\hspace*{3mm}{\tt CallShape} \\ 
\hspace*{3mm}{\tt  $\mid$ Name "-$>$" CallShape} \\ 
\hspace*{3mm}{\tt  $\mid$ CallShape ":=" Arg} \\ 
\end{tabular}

\begin{tabular}{l}
{\bf def92. CallShape ::= }\\ 
\hspace*{3mm}{\tt PrefixShape} \\ 
\hspace*{3mm}{\tt  $\mid$ DottedShape} \\ 
\hspace*{3mm}{\tt  $\mid$ InfixShape} \\ 
\end{tabular}

\begin{tabular}{l}
{\bf def93. PrefixShape ::= }\\ 
\hspace*{3mm}{\tt NameA {[}(Arglist){]}} \\ 
\end{tabular}

\begin{tabular}{l}
{\bf def94. DottedShape ::= }\\ 
\hspace*{3mm}{\tt NameB . NameA {[}(Arglist){]}} \\ 
\end{tabular}

\begin{tabular}{l}
{\bf def95. InfixShape ::= }\\ 
\hspace*{3mm}{\tt ArgA @ NameA {[}(ArgB){]}} \\ 
\end{tabular}

\begin{tabular}{l}
{\bf def96. Arglist ::= }\\ 
\hspace*{3mm}{\tt "(" Args ")"} \\ 
\end{tabular}

\begin{tabular}{l}
{\bf def97. Args ::= }\\ 
\hspace*{3mm}{\tt Arg** ","} \\ 
\end{tabular}

\begin{tabular}{l}
{\bf def98. Arg ::= }\\ 
\hspace*{3mm}{\tt Name {[}(":" {[}("!"){]} Type $\mid$ "==" Expr){]}} \\ 
\hspace*{3mm}{\tt  $\mid$ {[}(Name){]} "..."} \\ 
\end{tabular}

\begin{tabular}{l}
{\bf def99. ProcedureBody ::= }\\ 
\hspace*{3mm}{\tt StatementSeq} \\ 
\hspace*{3mm}{\tt  $\mid$ ArgList "=$>$" ProcedureBody} \\ 
\hspace*{3mm}{\tt  $\mid$ Arg "=$>$" ProcedureBody} \\ 
\end{tabular}

\begin{tabular}{l}
{\bf def100. VarDef ::= }\\ 
\hspace*{3mm}{\tt MarkedModifier ("const" $\mid$ "var") NameDecl} \\ 
\end{tabular}

\begin{tabular}{l}
{\bf def101. NameDecl ::= }\\ 
\hspace*{3mm}{\tt \begin{tabular}{l}
(OneDecl $\mid$ "(" OneDecl++ "," ")") \\ 
 {[}(("=" $\mid$ ":=") Expr){]} \\ 
\end{tabular}
} \\ 
\end{tabular}

\begin{tabular}{l}
{\bf def102. OneDecl ::= }\\ 
\hspace*{3mm}{\tt Name {[}(":" Type $\mid$ "..."){]}} \\ 
\end{tabular}

\begin{tabular}{l}
{\bf def103. ClassDef ::= }\\ 
\hspace*{3mm}{\tt \begin{tabular}{l}
"define" \\ 
 Modifier \\ 
 "class" \\ 
 Name \\ 
 {[}("extends" CommaExpr){]} \\ 
 ClassProperty** \\ 
 SEMI \\ 
 "enddefine" \\ 
\end{tabular}
} \\ 
\end{tabular}

\begin{tabular}{l}
{\bf def104. ClassProperty ::= }\\ 
\hspace*{3mm}{\tt Procedure} \\ 
\hspace*{3mm}{\tt  $\mid$ SlotDecl} \\ 
\hspace*{3mm}{\tt  $\mid$ VarDef} \\ 
\end{tabular}

\begin{tabular}{l}
{\bf def105. SlotDecl ::= }\\ 
\hspace*{3mm}{\tt \begin{tabular}{l}
("slot" $\mid$ "shared") \\ 
 Name \\ 
 {[}(":" Type){]} \\ 
 {[}("=" Expr){]} \\ 
 {[}("implements" Name2){]} \\ 
\end{tabular}
} \\ 
\end{tabular}

\begin{tabular}{l}
{\bf def106. UnitDef ::= }\\ 
\hspace*{3mm}{\tt \begin{tabular}{l}
"define" \\ 
 ("unit" $\mid$ "units") \\ 
 UnitDef++ \\ 
 "," \\ 
 "enddefine" \\ 
\end{tabular}
} \\ 
\end{tabular}

\begin{tabular}{l}
{\bf def107. UnitDef ::= }\\ 
\hspace*{3mm}{\tt Word {[}("=" Expr){]}} \\ 
\end{tabular}

\begin{tabular}{l}
{\bf def108. EnumDef ::= }\\ 
\hspace*{3mm}{\tt \begin{tabular}{l}
"define" \\ 
 "enum" \\ 
 {[}(Modifier){]} \\ 
 Name \\ 
 "=" \\ 
 Name++ \\ 
 "," \\ 
 "enddefine" \\ 
\end{tabular}
} \\ 
\end{tabular}

\begin{tabular}{l}
{\bf def109. Statement ::= }\\ 
\hspace*{3mm}{\tt Expr++ "," SEMI} \\ 
\end{tabular}

\begin{tabular}{l}
{\bf def110. StatementSeq ::= }\\ 
\hspace*{3mm}{\tt Statement*} \\ 
\end{tabular}

\begin{tabular}{l}
{\bf def111. Expr ::= }\\ 
\hspace*{3mm}{\tt Name} \\ 
\hspace*{3mm}{\tt  $\mid$ Hole} \\ 
\hspace*{3mm}{\tt  $\mid$ Literal} \\ 
\hspace*{3mm}{\tt  $\mid$ Expr "..."} \\ 
\hspace*{3mm}{\tt  $\mid$ PreOp Expr} \\ 
\hspace*{3mm}{\tt  $\mid$ Expr PostOp} \\ 
\hspace*{3mm}{\tt  $\mid$ Expr InOp Expr} \\ 
\hspace*{3mm}{\tt  $\mid$ Expr AssignOp Expr} \\ 
\hspace*{3mm}{\tt  $\mid$ "(" Expr ")"} \\ 
\hspace*{3mm}{\tt  $\mid$ Expr {[}("." DotExpr){]} {[}(OptExpr){]}} \\ 
\hspace*{3mm}{\tt  $\mid$ Expr {[}("@" DotExpr){]} {[}(OptExpr){]}} \\ 
\hspace*{3mm}{\tt  $\mid$ Expr "{[}" Expr "{]}"} \\ 
\hspace*{3mm}{\tt  $\mid$ "{[}" Expr "{]}"} \\ 
\hspace*{3mm}{\tt  $\mid$ LambdaExpr} \\ 
\hspace*{3mm}{\tt  $\mid$ "new" Name {[}("(" Expr ")"){]}} \\ 
\hspace*{3mm}{\tt  $\mid$ "none" OptExpr} \\ 
\hspace*{3mm}{\tt  $\mid$ "one" Expr} \\ 
\hspace*{3mm}{\tt  $\mid$ "once" Expr} \\ 
\hspace*{3mm}{\tt  $\mid$ "throw" Expr} \\ 
\hspace*{3mm}{\tt  $\mid$ "{[}" CommaExpr "{]}"} \\ 
\hspace*{3mm}{\tt  $\mid$ "\{" CommaExpr "\}"} \\ 
\hspace*{3mm}{\tt  $\mid$ QuoteExpr} \\ 
\hspace*{3mm}{\tt  $\mid$ TryCatchExpr} \\ 
\hspace*{3mm}{\tt  $\mid$ IfExpr} \\ 
\hspace*{3mm}{\tt  $\mid$ RepeatExpr} \\ 
\hspace*{3mm}{\tt  $\mid$ ForExpr} \\ 
\hspace*{3mm}{\tt  $\mid$ SwitchExpr} \\ 
\hspace*{3mm}{\tt  $\mid$ VarDef} \\ 
\hspace*{3mm}{\tt  $\mid$ ProcedureDef} \\ 
\end{tabular}

\begin{tabular}{l}
{\bf def112. OptExpr ::= }\\ 
\hspace*{3mm}{\tt {[}(Expr){]}} \\ 
\end{tabular}

\begin{tabular}{l}
{\bf def113. CommaExpr ::= }\\ 
\hspace*{3mm}{\tt Expr++ ","} \\ 
\end{tabular}

\begin{tabular}{l}
{\bf def114. OptCommaExpr ::= }\\ 
\hspace*{3mm}{\tt {[}(CommaExpr){]}} \\ 
\end{tabular}

\begin{tabular}{l}
{\bf def115. Hole ::= }\\ 
\hspace*{3mm}{\tt "?" {[}(Integer){]}} \\ 
\end{tabular}

\begin{tabular}{l}
{\bf def116. Literal ::= }\\ 
\hspace*{3mm}{\tt StringLiteral} \\ 
\hspace*{3mm}{\tt  $\mid$ NumberLiteral} \\ 
\hspace*{3mm}{\tt  $\mid$ ReservedLiteral} \\ 
\hspace*{3mm}{\tt  $\mid$ CharacterLiteral} \\ 
\end{tabular}

\begin{tabular}{l}
{\bf def117. PostOp ::= }\\ 
\hspace*{3mm}{\tt LexicalPostfixOperator} \\ 
\hspace*{3mm}{\tt  $\mid$ "@" DotExpr} \\ 
\end{tabular}

\begin{tabular}{l}
{\bf def118. InOp ::= }\\ 
\hspace*{3mm}{\tt LexicalInfixOperator} \\ 
\hspace*{3mm}{\tt  $\mid$ "@" DotExpr} \\ 
\end{tabular}

\begin{tabular}{l}
{\bf def119. DotExpr ::= }\\ 
\hspace*{3mm}{\tt Name} \\ 
\hspace*{3mm}{\tt  $\mid$ "(" Expr ")"} \\ 
\hspace*{3mm}{\tt  $\mid$ "return"} \\ 
\hspace*{3mm}{\tt  $\mid$ "throw"} \\ 
\hspace*{3mm}{\tt  $\mid$ "new"} \\ 
\end{tabular}

\begin{tabular}{l}
{\bf def120. LambdaExpr ::= }\\ 
\hspace*{3mm}{\tt "(" Args "=$>$" StatementSeq ")"} \\ 
\hspace*{3mm}{\tt  $\mid$ "fun" Args "=$>$" StatementSeq "endfun"} \\ 
\end{tabular}

\begin{tabular}{l}
{\bf def121. QuoteExpr ::= }\\ 
\hspace*{3mm}{\tt "`" QuotedItems "`"} \\ 
\end{tabular}

\begin{tabular}{l}
{\bf def122. QuotedItems ::= }\\ 
\hspace*{3mm}{\tt QuotedItem*} \\ 
\end{tabular}

\begin{tabular}{l}
{\bf def123. QuotedItem ::= }\\ 
\hspace*{3mm}{\tt Word} \\ 
\hspace*{3mm}{\tt  $\mid$ "\{" QuotedItems "\}"} \\ 
\hspace*{3mm}{\tt  $\mid$ "(" Expr ")"} \\ 
\hspace*{3mm}{\tt  $\mid$ "$^\wedge$" Word} \\ 
\hspace*{3mm}{\tt  $\mid$ "$^\wedge$$^\wedge$" Word} \\ 
\hspace*{3mm}{\tt  $\mid$ Literal} \\ 
\end{tabular}

\begin{tabular}{l}
{\bf def124. TryCatchExpr ::= }\\ 
\hspace*{3mm}{\tt "try" StatementSeq CatchSeq "endtry"} \\ 
\end{tabular}

\begin{tabular}{l}
{\bf def125. CatchSeq ::= }\\ 
\hspace*{3mm}{\tt \begin{tabular}{l}
("catch" Arglist {[}("as"){]} StatementSeq)** \\ 
\end{tabular}
} \\ 
\end{tabular}

\begin{tabular}{l}
{\bf def126. IfExpr ::= }\\ 
\hspace*{3mm}{\tt LongIfExpr} \\ 
\hspace*{3mm}{\tt  $\mid$ ShortIfExpr} \\ 
\end{tabular}

\begin{tabular}{l}
{\bf def127. ShortIfExpr ::= }\\ 
\hspace*{3mm}{\tt \begin{tabular}{l}
"(" \\ 
 Expr \\ 
 "$\mid$" \\ 
 CommaExpr \\ 
 ("$\mid$:" Expr "$\mid$" CommaExpr)* \\ 
 {[}("$\mid$" CommaExpr){]} \\ 
 ")" \\ 
\end{tabular}
} \\ 
\end{tabular}

\begin{tabular}{l}
{\bf def128. LongIfExpr ::= }\\ 
\hspace*{3mm}{\tt \begin{tabular}{l}
("if" $\mid$ "unless") \\ 
 Expr \\ 
 Then \\ 
 StatementSeq \\ 
 ("elseif" Expr Then StatementSeq)* \\ 
 {[}("else" StatementSeq){]} \\ 
 ("endif" $\mid$ "endunless") \\ 
\end{tabular}
} \\ 
\end{tabular}

\begin{tabular}{l}
{\bf def129. Then ::= }\\ 
\hspace*{3mm}{\tt "then"} \\ 
\hspace*{3mm}{\tt  $\mid$ "do"} \\ 
\end{tabular}

\begin{tabular}{l}
{\bf def130. RepeatExpr ::= }\\ 
\hspace*{3mm}{\tt \begin{tabular}{l}
"repeat" \\ 
 StatementSeq1 \\ 
 {[}(RepeatTest {[}("do" StatementSeq2){]}){]} \\ 
 "endrepeat" \\ 
\end{tabular}
} \\ 
\end{tabular}

\begin{tabular}{l}
{\bf def131. RepeatTest ::= }\\ 
\hspace*{3mm}{\tt ("while" $\mid$ "until") Expr1 {[}("then" Expr2){]}} \\ 
\end{tabular}

\begin{tabular}{l}
{\bf def132. ForExpr ::= }\\ 
\hspace*{3mm}{\tt \begin{tabular}{l}
"for" \\ 
 ForControls \\ 
 {[}("do" StatementSeq){]} \\ 
 {[}("finally" StatementSeq){]} \\ 
 "endfor" \\ 
\end{tabular}
} \\ 
\end{tabular}

\begin{tabular}{l}
{\bf def133. ForControls ::= }\\ 
\hspace*{3mm}{\tt ForControl} \\ 
\hspace*{3mm}{\tt  $\mid$ ForControls "also" ForControl} \\ 
\hspace*{3mm}{\tt  $\mid$ ForControls ForCondition} \\ 
\end{tabular}

\begin{tabular}{l}
{\bf def134. ForControl ::= }\\ 
\hspace*{3mm}{\tt ForBinding} \\ 
\hspace*{3mm}{\tt  $\mid$ ForCondition} \\ 
\end{tabular}

\begin{tabular}{l}
{\bf def135. ForCondition ::= }\\ 
\hspace*{3mm}{\tt \begin{tabular}{l}
("while" $\mid$ "until") \\ 
 Expr \\ 
 {[}(("then" $\mid$ "else") CommaExpr){]} \\ 
\end{tabular}
} \\ 
\end{tabular}

\begin{tabular}{l}
{\bf def136. ForBinding ::= }\\ 
\hspace*{3mm}{\tt ForName "in" Expr} \\ 
\hspace*{3mm}{\tt  $\mid$ \begin{tabular}{l}
ForName \\ 
 "from" \\ 
 Expr \\ 
 {[}(("to" $\mid$ "downto") Expr){]} \\ 
\end{tabular}
} \\ 
\hspace*{3mm}{\tt  $\mid$ ForName "to" Expr} \\ 
\end{tabular}

\begin{tabular}{l}
{\bf def137. ForName ::= }\\ 
\hspace*{3mm}{\tt Name {[}(TypeWord Type){]}} \\ 
\end{tabular}

\begin{tabular}{l}
{\bf def138. SwitchExpr ::= }\\ 
\hspace*{3mm}{\tt "switch" Expr Case+ "endswitch"} \\ 
\end{tabular}

\begin{tabular}{l}
{\bf def139. Case ::= }\\ 
\hspace*{3mm}{\tt ("case" Expr)+ "then" StatementSeq} \\ 
\hspace*{3mm}{\tt  $\mid$ "else" StatementSeq} \\ 
\end{tabular}

\begin{tabular}{l}
{\bf def140. Type ::= }\\ 
\hspace*{3mm}{\tt Expr} \\ 
\end{tabular}

\begin{tabular}{l}
{\bf def141. Pathname ::= }\\ 
\hspace*{3mm}{\tt \begin{tabular}{l}
{[}(Scheme){]} \\ 
 {[}(Root){]} \\ 
 {[}(Directories){]} \\ 
 {[}(Name){]} \\ 
 {[}(Suffix){]} \\ 
 {[}(Type){]} \\ 
\end{tabular}
} \\ 
\end{tabular}

\begin{tabular}{l}
{\bf def142. Scheme ::= }\\ 
\hspace*{3mm}{\tt Word ":"} \\ 
\end{tabular}

\begin{tabular}{l}
{\bf def143. Root ::= }\\ 
\hspace*{3mm}{\tt "//" Dotted} \\ 
\end{tabular}

\begin{tabular}{l}
{\bf def144. Directories ::= }\\ 
\hspace*{3mm}{\tt {[}("/"){]} (Dotted "/")++} \\ 
\end{tabular}

\begin{tabular}{l}
{\bf def145. Name ::= }\\ 
\hspace*{3mm}{\tt Word} \\ 
\end{tabular}

\begin{tabular}{l}
{\bf def146. Suffix ::= }\\ 
\hspace*{3mm}{\tt "." Word} \\ 
\end{tabular}

\begin{tabular}{l}
{\bf def147. Type ::= }\\ 
\hspace*{3mm}{\tt ";" Word} \\ 
\end{tabular}

\begin{tabular}{l}
{\bf def148. Dotted ::= }\\ 
\hspace*{3mm}{\tt Word++ "."} \\ 
\end{tabular}



\end{document}
